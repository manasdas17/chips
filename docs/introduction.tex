\chapter{Introduction}
\section{A new approach to device design}

Traditionaly, the tool of choice for digital devices is a Hardware Description
Language (HDL), the most common being Verilog and VHDL. These languages provide
a reasonably rich environment for modeling and simulating hardware,
but only a limitied subset of the langauge can be realised in a digital
device (synthesised). 

A software designer would typically implement a function in an imperative
style using loops, branches and subprocedures, but hardware models written in
an imperative style cannot be synthesised.

Synthesisable designs require a different approach. Digital device designers
must work at the Register Transfer Level (RTL). The primitive elements of an
RTL design are clocked memory elements (registers) and combinational
logic elements. A typlical synthesis tool would be able to infer boolean logic,
addition, subtraction, multiplexing and bit manipulation from HDL code written
in a very specific style.

An RTL designer has to work at a low level of abstraction. In practical terms
this means that a designer has to do more of the work themselves.

\begin{enumerate}

\item
A designer is reponsible for designing their own interfaces to the outside
world.

\item
The designer is responsible for clock to clock timing, manualy balancing
propogation delays between clocked elements to achieve high performance.

\item
A designer has to provide their own mechanism to synchronise and pass
data bwetween concurrent computational elements (by implementing a bus with
control and handshaking signals).

\item
A designer has to provide their own mechanism to control the flow of execution
within a computational element (usually by manually coding a finite state
machine). 

\item
The primitive elements are primitive.

\end{enumerate}

This is where Python Streams comes in. In Python Streams, there is no
synthesiseable subset, but a standalone synthesisable language built on top of
Python. Python streams allows designers to work at a higher level of
abstraction. It does a lot more of the work for you.

\begin{enumerate}

\item Python streams provides a suite of device interfaces including portio,
uart, usb and ethernet.

\item Synthesisable RTL code is generated automatically by the tool. Clocks,
resets, and clock to clock timing are all taken care of behind the scenes.

\item Python Streams provides a simple method to synchronise concurrent
elements, and to pass data between them - streams. The tool automatically
generates interconnect busses and handshaking signals behind the scenes.

\item Python streams provides imperative style sequeneces branches and loop. The
tool automatically generates state machines, or highly optimized soft-core
processors behind the scenes.

\item The primitive elements are not so primitive. Common constructs such as
counters, lookup tables, ROMS and RAMS are invoked with a single keyword and a
few parameters. Python Streams also provides a richer set of arithmetic
operators including fully synthesiseable division and multiplication.

\end{enumerate}

\section{A language within a language}

Python Streams is a python library, just an add-on to Python which is no more
or less than a programming language. The Python Streams library provides an
Application Programmers Interface (API) to a suite of hardware design functions.

The Python Streams library can also be considered a language in its own right,
The Python language itself provides statements which are executed on your
own computer. The Python Streams provides an alternative language, statements
which are executed on the target device.
