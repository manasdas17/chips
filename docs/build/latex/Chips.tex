% Generated by Sphinx.
\documentclass[letterpaper,10pt,english]{manual}
\usepackage[utf8]{inputenc}
\usepackage[T1]{fontenc}
\usepackage{babel}
\usepackage{times}
\usepackage[Bjarne]{fncychap}
\usepackage{longtable}
\usepackage{sphinx}


\title{Chips Documentation}
\date{April 14, 2011}
\release{0.1}
\author{Jonathan P Dawson}
\newcommand{\sphinxlogo}{}
\renewcommand{\releasename}{Release}
\makeindex
\makemodindex

\makeatletter
\def\PYG@reset{\let\PYG@it=\relax \let\PYG@bf=\relax%
    \let\PYG@ul=\relax \let\PYG@tc=\relax%
    \let\PYG@bc=\relax \let\PYG@ff=\relax}
\def\PYG@tok#1{\csname PYG@tok@#1\endcsname}
\def\PYG@toks#1+{\ifx\relax#1\empty\else%
    \PYG@tok{#1}\expandafter\PYG@toks\fi}
\def\PYG@do#1{\PYG@bc{\PYG@tc{\PYG@ul{%
    \PYG@it{\PYG@bf{\PYG@ff{#1}}}}}}}
\def\PYG#1#2{\PYG@reset\PYG@toks#1+\relax+\PYG@do{#2}}

\def\PYG@tok@gd{\def\PYG@tc##1{\textcolor[rgb]{0.63,0.00,0.00}{##1}}}
\def\PYG@tok@gu{\let\PYG@bf=\textbf\def\PYG@tc##1{\textcolor[rgb]{0.50,0.00,0.50}{##1}}}
\def\PYG@tok@gt{\def\PYG@tc##1{\textcolor[rgb]{0.00,0.25,0.82}{##1}}}
\def\PYG@tok@gs{\let\PYG@bf=\textbf}
\def\PYG@tok@gr{\def\PYG@tc##1{\textcolor[rgb]{1.00,0.00,0.00}{##1}}}
\def\PYG@tok@cm{\let\PYG@it=\textit\def\PYG@tc##1{\textcolor[rgb]{0.25,0.50,0.56}{##1}}}
\def\PYG@tok@vg{\def\PYG@tc##1{\textcolor[rgb]{0.73,0.38,0.84}{##1}}}
\def\PYG@tok@m{\def\PYG@tc##1{\textcolor[rgb]{0.13,0.50,0.31}{##1}}}
\def\PYG@tok@mh{\def\PYG@tc##1{\textcolor[rgb]{0.13,0.50,0.31}{##1}}}
\def\PYG@tok@cs{\def\PYG@tc##1{\textcolor[rgb]{0.25,0.50,0.56}{##1}}\def\PYG@bc##1{\colorbox[rgb]{1.00,0.94,0.94}{##1}}}
\def\PYG@tok@ge{\let\PYG@it=\textit}
\def\PYG@tok@vc{\def\PYG@tc##1{\textcolor[rgb]{0.73,0.38,0.84}{##1}}}
\def\PYG@tok@il{\def\PYG@tc##1{\textcolor[rgb]{0.13,0.50,0.31}{##1}}}
\def\PYG@tok@go{\def\PYG@tc##1{\textcolor[rgb]{0.19,0.19,0.19}{##1}}}
\def\PYG@tok@cp{\def\PYG@tc##1{\textcolor[rgb]{0.00,0.44,0.13}{##1}}}
\def\PYG@tok@gi{\def\PYG@tc##1{\textcolor[rgb]{0.00,0.63,0.00}{##1}}}
\def\PYG@tok@gh{\let\PYG@bf=\textbf\def\PYG@tc##1{\textcolor[rgb]{0.00,0.00,0.50}{##1}}}
\def\PYG@tok@ni{\let\PYG@bf=\textbf\def\PYG@tc##1{\textcolor[rgb]{0.84,0.33,0.22}{##1}}}
\def\PYG@tok@nl{\let\PYG@bf=\textbf\def\PYG@tc##1{\textcolor[rgb]{0.00,0.13,0.44}{##1}}}
\def\PYG@tok@nn{\let\PYG@bf=\textbf\def\PYG@tc##1{\textcolor[rgb]{0.05,0.52,0.71}{##1}}}
\def\PYG@tok@no{\def\PYG@tc##1{\textcolor[rgb]{0.38,0.68,0.84}{##1}}}
\def\PYG@tok@na{\def\PYG@tc##1{\textcolor[rgb]{0.25,0.44,0.63}{##1}}}
\def\PYG@tok@nb{\def\PYG@tc##1{\textcolor[rgb]{0.00,0.44,0.13}{##1}}}
\def\PYG@tok@nc{\let\PYG@bf=\textbf\def\PYG@tc##1{\textcolor[rgb]{0.05,0.52,0.71}{##1}}}
\def\PYG@tok@nd{\let\PYG@bf=\textbf\def\PYG@tc##1{\textcolor[rgb]{0.33,0.33,0.33}{##1}}}
\def\PYG@tok@ne{\def\PYG@tc##1{\textcolor[rgb]{0.00,0.44,0.13}{##1}}}
\def\PYG@tok@nf{\def\PYG@tc##1{\textcolor[rgb]{0.02,0.16,0.49}{##1}}}
\def\PYG@tok@si{\let\PYG@it=\textit\def\PYG@tc##1{\textcolor[rgb]{0.44,0.63,0.82}{##1}}}
\def\PYG@tok@s2{\def\PYG@tc##1{\textcolor[rgb]{0.25,0.44,0.63}{##1}}}
\def\PYG@tok@vi{\def\PYG@tc##1{\textcolor[rgb]{0.73,0.38,0.84}{##1}}}
\def\PYG@tok@nt{\let\PYG@bf=\textbf\def\PYG@tc##1{\textcolor[rgb]{0.02,0.16,0.45}{##1}}}
\def\PYG@tok@nv{\def\PYG@tc##1{\textcolor[rgb]{0.73,0.38,0.84}{##1}}}
\def\PYG@tok@s1{\def\PYG@tc##1{\textcolor[rgb]{0.25,0.44,0.63}{##1}}}
\def\PYG@tok@gp{\let\PYG@bf=\textbf\def\PYG@tc##1{\textcolor[rgb]{0.78,0.36,0.04}{##1}}}
\def\PYG@tok@sh{\def\PYG@tc##1{\textcolor[rgb]{0.25,0.44,0.63}{##1}}}
\def\PYG@tok@ow{\let\PYG@bf=\textbf\def\PYG@tc##1{\textcolor[rgb]{0.00,0.44,0.13}{##1}}}
\def\PYG@tok@sx{\def\PYG@tc##1{\textcolor[rgb]{0.78,0.36,0.04}{##1}}}
\def\PYG@tok@bp{\def\PYG@tc##1{\textcolor[rgb]{0.00,0.44,0.13}{##1}}}
\def\PYG@tok@c1{\let\PYG@it=\textit\def\PYG@tc##1{\textcolor[rgb]{0.25,0.50,0.56}{##1}}}
\def\PYG@tok@kc{\let\PYG@bf=\textbf\def\PYG@tc##1{\textcolor[rgb]{0.00,0.44,0.13}{##1}}}
\def\PYG@tok@c{\let\PYG@it=\textit\def\PYG@tc##1{\textcolor[rgb]{0.25,0.50,0.56}{##1}}}
\def\PYG@tok@mf{\def\PYG@tc##1{\textcolor[rgb]{0.13,0.50,0.31}{##1}}}
\def\PYG@tok@err{\def\PYG@bc##1{\fcolorbox[rgb]{1.00,0.00,0.00}{1,1,1}{##1}}}
\def\PYG@tok@kd{\let\PYG@bf=\textbf\def\PYG@tc##1{\textcolor[rgb]{0.00,0.44,0.13}{##1}}}
\def\PYG@tok@ss{\def\PYG@tc##1{\textcolor[rgb]{0.32,0.47,0.09}{##1}}}
\def\PYG@tok@sr{\def\PYG@tc##1{\textcolor[rgb]{0.14,0.33,0.53}{##1}}}
\def\PYG@tok@mo{\def\PYG@tc##1{\textcolor[rgb]{0.13,0.50,0.31}{##1}}}
\def\PYG@tok@mi{\def\PYG@tc##1{\textcolor[rgb]{0.13,0.50,0.31}{##1}}}
\def\PYG@tok@kn{\let\PYG@bf=\textbf\def\PYG@tc##1{\textcolor[rgb]{0.00,0.44,0.13}{##1}}}
\def\PYG@tok@o{\def\PYG@tc##1{\textcolor[rgb]{0.40,0.40,0.40}{##1}}}
\def\PYG@tok@kr{\let\PYG@bf=\textbf\def\PYG@tc##1{\textcolor[rgb]{0.00,0.44,0.13}{##1}}}
\def\PYG@tok@s{\def\PYG@tc##1{\textcolor[rgb]{0.25,0.44,0.63}{##1}}}
\def\PYG@tok@kp{\def\PYG@tc##1{\textcolor[rgb]{0.00,0.44,0.13}{##1}}}
\def\PYG@tok@w{\def\PYG@tc##1{\textcolor[rgb]{0.73,0.73,0.73}{##1}}}
\def\PYG@tok@kt{\def\PYG@tc##1{\textcolor[rgb]{0.56,0.13,0.00}{##1}}}
\def\PYG@tok@sc{\def\PYG@tc##1{\textcolor[rgb]{0.25,0.44,0.63}{##1}}}
\def\PYG@tok@sb{\def\PYG@tc##1{\textcolor[rgb]{0.25,0.44,0.63}{##1}}}
\def\PYG@tok@k{\let\PYG@bf=\textbf\def\PYG@tc##1{\textcolor[rgb]{0.00,0.44,0.13}{##1}}}
\def\PYG@tok@se{\let\PYG@bf=\textbf\def\PYG@tc##1{\textcolor[rgb]{0.25,0.44,0.63}{##1}}}
\def\PYG@tok@sd{\let\PYG@it=\textit\def\PYG@tc##1{\textcolor[rgb]{0.25,0.44,0.63}{##1}}}

\def\PYGZbs{\char`\\}
\def\PYGZus{\char`\_}
\def\PYGZob{\char`\{}
\def\PYGZcb{\char`\}}
\def\PYGZca{\char`\^}
% for compatibility with earlier versions
\def\PYGZat{@}
\def\PYGZlb{[}
\def\PYGZrb{]}
\makeatother

\begin{document}

\maketitle
\tableofcontents
\hypertarget{--doc-index}{}


Contents:

\resetcurrentobjects
\hypertarget{--doc-introduction/index}{}

\chapter{Introduction}

The Chips library gives Python the ability to design, simulate and realise
digital devices such as FPGAs. Chips provides a simple yet powerful suite of
primitive components, \emph{Streams}, \emph{Processes} and \emph{Sinks} that can be
succinctly combined to form \emph{Chips}. The \emph{Chips} library can automatically
convert \emph{Streams}, \emph{Processes} and \emph{Sinks} into a Hardware Description
Language, which can be synthesised into real hardware.

Python programs cannot themselves be converted into real hardware, but it is
possible to programmatically generate which construct \emph{Chips}, which can
in-turn be converted into hardware. When combined with the extensive
libraries already supported by Python, such as NumPy and SciPy, Python and
Chips make the ideal design and verification environment.


\section{A new approach to device design}

Traditionally, the tool of choice for digital devices is a Hardware
Description Language (HDL), the most common being Verilog and VHDL. These
languages provide a reasonably rich environment for modeling and simulating
hardware, but only a limited subset of the language can be realised in a
digital device (synthesised).

While a software designer would typically implement a function in an
imperative style using loops, branches and sub procedures; a hardware model
written in an imperative style cannot be synthesised.

Synthesizable designs require a different approach. Digital device designers
must work at the Register Transfer Level (RTL). The primitive elements of an
RTL design are clocked memory elements (registers) and combinational logic
elements. A typical synthesis tool would be able to infer boolean logic,
addition, subtraction, multiplexing and bit manipulation from HDL code
written in a very specific style.

An RTL designer has to work at a low level of abstraction. In practical
terms this means that a designer has to do more of the work themselves.
\begin{enumerate}
\item {} 
A designer is responsible for designing their own interfaces to
the outside world.

\item {} 
The designer is responsible for clock to clock timing, manually
balancing propagation delays between clocked elements to achieve
high performance.

\item {} 
A designer has to provide their own mechanism to synchronise and
pass data between concurrent computational elements (by
implementing a bus with control and handshaking signals).

\item {} 
A designer has to provide their own mechanism to control the flow
of execution within a computational element (usually by manually
coding a finite state machine).

\item {} 
The primitive elements are primitive. Synthesis tools provide
limited support for multiplication, and division is not usually
supported at all.

\end{enumerate}

This is where \emph{Python Chips} comes in. In \emph{Python Chips}, there is no
synthesizable subset, but a standalone synthesizable language built on top
of Python. \emph{Python Chips} allows designers to work at a higher level of
abstraction. It does a lot more of the work for you.
\begin{enumerate}
\item {} 
\emph{Python Chips} provides a suite of device interfaces including
I/O ports and USARTs.

\item {} 
Synthesizable RTL code is generated automatically by the tool.
Clocks, resets, and clock to clock timing are all taken care of
behind the scenes.

\item {} 
\emph{Python Chips} provides a simple method to synchronise concurrent
elements, and to pass data between them - streams. The tool
automatically generates interconnect buses and handshaking
signals behind the scenes.

\item {} 
\emph{Python Chips} provides processes with imperative style
sequences branches and loop. The tool automatically generates
state machines, or highly optimized soft-core processors behind
the scenes.

\item {} 
The primitive elements are not so primitive. Common constructs
such as counters, lookup tables, ROMS and RAMS are invoked with a
single keyword and a few parameters. \emph{Python Chips} also provides
a richer set of arithmetic operators including fully
synthesizable division and multiplication.

\end{enumerate}


\section{A language within a language}

\emph{Python Chips} is a python library, just an add-on to Python which is no
more or less than a programming language. The \emph{Python Chips} library
provides an Application Programmers Interface (API) to a suite of hardware
design functions.

The \emph{Python Chips} library can also be considered a language in its own
right, The Python language itself provides statements which are executed on
your own computer. The \emph{Python Chips} provides an alternative language,
statements which are executed on the target device.

\resetcurrentobjects
\hypertarget{--doc-tutorial/index}{}

\chapter{Tutorial}


\section{Learn Python}

In order to make any real use of the \emph{Chips} library you will need to be
familiar with the basics of Python. The \href{http://docs.python.org/tut}{python tutorial} is a good place
to start.


\section{Install Chips}


\subsection{Windows}
\begin{enumerate}
\item {} 
First \href{http://python.org/download}{install python}. You need Python 2.6 or later, but not Python 3.

\item {} 
Then install the chips library from the \href{http://github.com/dawsonjon/chips}{windows installer}.

\end{enumerate}


\subsection{Linux}
\begin{enumerate}
\item {} 
First \href{http://python.org/download}{install python}. You need Python 2.6 or later, but not Python 3.

\item {} 
Then install the chips library from the \href{http://github.com/dawsonjon/chips}{source distribution}:

\begin{Verbatim}[commandchars=@\[\]]
desktop:@textasciitilde[]@$ tar -zxf Chips-0.1.tar.gz
desktop:@textasciitilde[]@$ cd Chips-0.1
desktop:@textasciitilde[]@$ python setup.py install @#run as root
\end{Verbatim}

\end{enumerate}


\section{First Simulations}

Once you have python and chips all set up, you can start with some simple
examples. This one counts to 10 repeatedly:

\begin{Verbatim}[commandchars=\\\{\}]
\PYG{k+kn}{from} \PYG{n+nn}{streams} \PYG{k+kn}{import} \PYG{o}{*}

\PYG{c}{\#create a chip model}
\PYG{n}{my\PYGZus{}chip} \PYG{o}{=} \PYG{n}{Chip}\PYG{p}{(}
  \PYG{n}{Console}\PYG{p}{(}
    \PYG{n}{Printer}\PYG{p}{(}
        \PYG{n}{Counter}\PYG{p}{(}\PYG{l+m+mi}{0}\PYG{p}{,} \PYG{l+m+mi}{10}\PYG{p}{,} \PYG{l+m+mi}{1}\PYG{p}{)}\PYG{p}{,}
    \PYG{p}{)}\PYG{p}{,}
  \PYG{p}{)}\PYG{p}{,}
\PYG{p}{)}

\PYG{c}{\#run a simulation}
\PYG{n}{my\PYGZus{}chip}\PYG{o}{.}\PYG{n}{reset}\PYG{p}{(}\PYG{p}{)}
\PYG{n}{my\PYGZus{}chip}\PYG{o}{.}\PYG{n}{execute}\PYG{p}{(}\PYG{l+m+mi}{100}\PYG{p}{)}
\end{Verbatim}

The example can be broken down as follows:
\begin{itemize}
\item {} 
\code{from stream import *} adds the basic features of the streams
library to the local namespace.

\item {} 
A \emph{Chip} models a target device. You need to tell it what the outputs
(\emph{sinks}) are, but it will work out what the inputs are by itself. In
this case the only \emph{sink} is the \emph{Console}.

\item {} 
A \emph{Console} is a sink that outputs a stream of data to the console.
The only argument it needs is the data stream, \emph{Printer}.

\item {} 
A \emph{Printer} is a \emph{stream} object that represents a stream of data in
decimal format as a string of ASCII characters. A \emph{Printer} is not a
source of data in itself, it transforms a stream of data that you
supply, the \emph{Counter}.

\item {} 
The \emph{Counter} is a fundamental data stream. It accepts three
arguments: start, stop and step. The \emph{Counter} will yield a stream of
data counting from \emph{start} to \emph{stop} in \emph{step} increments.

\end{itemize}


\section{Hello World}

No language would be complete without a ``hello world'' example:

\begin{Verbatim}[commandchars=\\\{\}]
\PYG{k+kn}{from} \PYG{n+nn}{chips} \PYG{k+kn}{import} \PYG{o}{*}

\PYG{c}{\#convert string into a sequence of characters}
\PYG{n}{hello\PYGZus{}world} \PYG{o}{=} \PYG{n+nb}{tuple}\PYG{p}{(}\PYG{p}{(}\PYG{n+nb}{ord}\PYG{p}{(}\PYG{n}{i}\PYG{p}{)} \PYG{k}{for} \PYG{n}{i} \PYG{o+ow}{in} \PYG{l+s}{"}\PYG{l+s}{hello world}\PYG{l+s+se}{\PYGZbs{}n}\PYG{l+s}{"}\PYG{p}{)}\PYG{p}{)}

\PYG{n}{my\PYGZus{}chip} \PYG{o}{=} \PYG{n}{Chip}\PYG{p}{(}
    \PYG{n}{Console}\PYG{p}{(}
        \PYG{n}{Sequence}\PYG{p}{(}\PYG{o}{*}\PYG{n}{hello\PYGZus{}world}\PYG{p}{)}\PYG{p}{,}
    \PYG{p}{)}
\PYG{p}{)}

\PYG{c}{\#run a simulation}
\PYG{n}{my\PYGZus{}chip}\PYG{o}{.}\PYG{n}{reset}\PYG{p}{(}\PYG{p}{)}
\PYG{n}{my\PYGZus{}chip}\PYG{o}{.}\PYG{n}{execute}\PYG{p}{(}\PYG{l+m+mi}{100}\PYG{p}{)}
\end{Verbatim}

In this example we have made only a few changes:
\begin{itemize}
\item {} 
\code{hello\_world = tuple((ord(i) for i in "hello world\textbackslash{}n"))} creates a tuple
containing the numeric values of the ascii characters in a string.

\item {} 
This example introduces a new stream, the \emph{Sequence}. The \emph{Sequence}
stream outputs each of its arguments in turn, when the arguments are
exhausted, the process repeats.

\item {} 
A \emph{Printer} is  \emph{stream} is not needed in this example since the stream is
already a sequence of ASCII values.

\end{itemize}


\section{Generating VHDL}

Now lets consider how the ``hello world'' example could be implemented in an
actual device. A first step to implementing a device would be to generate a
VHDL model:

\begin{Verbatim}[commandchars=\\\{\}]
\PYG{k+kn}{from} \PYG{n+nn}{chips} \PYG{k+kn}{import} \PYG{o}{*}
\PYG{k+kn}{from} \PYG{n+nn}{chips.VHDL\PYGZus{}plugin} \PYG{k+kn}{import} \PYG{n}{Plugin}

\PYG{c}{\#convert string into a sequence of characters}
\PYG{n}{hello\PYGZus{}world} \PYG{o}{=} \PYG{n+nb}{tuple}\PYG{p}{(}\PYG{p}{(}\PYG{n+nb}{ord}\PYG{p}{(}\PYG{n}{i}\PYG{p}{)} \PYG{k}{for} \PYG{n}{i} \PYG{o+ow}{in} \PYG{l+s}{"}\PYG{l+s}{hello world}\PYG{l+s+se}{\PYGZbs{}n}\PYG{l+s}{"}\PYG{p}{)}\PYG{p}{)}

\PYG{n}{my\PYGZus{}chip} \PYG{o}{=} \PYG{n}{Chip}\PYG{p}{(}
    \PYG{n}{Console}\PYG{p}{(}
        \PYG{n}{Sequence}\PYG{p}{(}\PYG{o}{*}\PYG{n}{hello\PYGZus{}world}\PYG{p}{)}\PYG{p}{,}
    \PYG{p}{)}
\PYG{p}{)}

\PYG{c}{\#generate a VHDL model}
\PYG{n}{code\PYGZus{}generator} \PYG{o}{=} \PYG{n}{Plugin}\PYG{p}{(}\PYG{n}{project\PYGZus{}name}\PYG{o}{=}\PYG{l+s}{"}\PYG{l+s}{hello world}\PYG{l+s}{"}\PYG{p}{)}
\PYG{n}{my\PYGZus{}chip}\PYG{o}{.}\PYG{n}{write\PYGZus{}code}\PYG{p}{(}\PYG{n}{code\PYGZus{}generator}\PYG{p}{)}
\end{Verbatim}

The \emph{Chips} library uses plugins to generate output code from models. This
means that new code generators can be added to Chips without having to
change the way that hardware is designed and simulated. At present, chips
supports C++ and VHDL code generation, but it is VHDL code that allows
{\color{red}\bfseries{}*}Chips'' to be synthesised.

The VHDL code generation plugin is found in \code{chips.VHDL\_plugin} If you run
this example you should find hello\_world.vhd has been generated. You can
run this code in an external vhdl simulator, but you won't be able to
synthesise it into a device because real hardware devices don't have a
concept of a \emph{Console}.

To make this example synthesise, we need to write the characters to some
realisable hardware interface. The \emph{Chips} library provides a \emph{SerialOut}
sink, this provides a simple way to direct the stream of characters to a
serial port:

\begin{Verbatim}[commandchars=\\\{\}]
\PYG{k+kn}{from} \PYG{n+nn}{chips} \PYG{k+kn}{import} \PYG{o}{*}
\PYG{k+kn}{from} \PYG{n+nn}{chips.VHDL\PYGZus{}plugin} \PYG{k+kn}{import} \PYG{n}{Plugin}

\PYG{c}{\#convert string into a sequence of characters}
\PYG{n}{hello\PYGZus{}world} \PYG{o}{=} \PYG{n+nb}{tuple}\PYG{p}{(}\PYG{p}{(}\PYG{n+nb}{ord}\PYG{p}{(}\PYG{n}{i}\PYG{p}{)} \PYG{k}{for} \PYG{n}{i} \PYG{o+ow}{in} \PYG{l+s}{"}\PYG{l+s}{hello world}\PYG{l+s+se}{\PYGZbs{}n}\PYG{l+s}{"}\PYG{p}{)}\PYG{p}{)}

\PYG{n}{my\PYGZus{}chip} \PYG{o}{=} \PYG{n}{Chip}\PYG{p}{(}
    \PYG{n}{SerialOut}\PYG{p}{(}
        \PYG{n}{Sequence}\PYG{p}{(}\PYG{o}{*}\PYG{n}{hello\PYGZus{}world}\PYG{p}{)}\PYG{p}{,}
    \PYG{p}{)}
\PYG{p}{)}

\PYG{c}{\#generate a VHDL model}
\PYG{n}{code\PYGZus{}generator} \PYG{o}{=} \PYG{n}{Plugin}\PYG{p}{(}\PYG{n}{project\PYGZus{}name}\PYG{o}{=}\PYG{l+s}{"}\PYG{l+s}{hello world}\PYG{l+s}{"}\PYG{p}{)}
\PYG{n}{my\PYGZus{}chip}\PYG{o}{.}\PYG{n}{write\PYGZus{}code}\PYG{p}{(}\PYG{n}{code\PYGZus{}generator}\PYG{p}{)}
\end{Verbatim}

Now you should have a hello\_world.vhd file that you can synthesise in a real
device. By default, SerialOut will assume that you are using a 50 MHz clock
and a baud rate of 115200. If you need something else you can use the
clock\_rate and baud\_rate arguments to specify what you need.


\section{More Streams}

So far we have seen three types of streams, \emph{Counter}, \emph{Sequence} and
\emph{Printer}. Chips provides a few more basic streams which you can read about
in the Language Reference Manual. It is also possible to combine streams
using arithmetic operators : \code{+, -, *, //, \%, \textless{}\textless{}, \textgreater{}\textgreater{}, \&, \textbar{}, \textasciicircum{}, ==, !=, \textless{},
\textless{}=, \textgreater{}, \textgreater{}=} on the whole they have the same (or very similar) meaning as
they do in Python except that they operate on streams of data.
\begin{quote}
\end{quote}


\section{Introducing Processes}

\resetcurrentobjects
\hypertarget{--doc-language\_reference/index}{}

\chapter{Chips Language Reference Manual}


\section{Chip}
\index{Chip (class in chips)}

\hypertarget{chips.Chip}{}\begin{classdesc}{Chip}{*args}
A Chip device containing streams, sinks and processes.

Typically a Chip is used to describe a single device. You need to provide
the Chip object with a list of all the sinks (device outputs). You don't
need to include any process, variables or streams. By analysing the sinks,
the chip can work out which processes and streams need to be included in
the device.

Example:

\begin{Verbatim}[commandchars=\\\{\}]
\PYG{n}{switches} \PYG{o}{=} \PYG{n}{InPort}\PYG{p}{(}\PYG{l+s}{"}\PYG{l+s}{SWITCHES}\PYG{l+s}{"}\PYG{p}{,} \PYG{l+m+mi}{8}\PYG{p}{)}
\PYG{n}{serial\PYGZus{}in} \PYG{o}{=} \PYG{n}{SerialIn}\PYG{p}{(}\PYG{l+s}{"}\PYG{l+s}{RX}\PYG{l+s}{"}\PYG{p}{)}
\PYG{n}{leds} \PYG{o}{=} \PYG{n}{OutPort}\PYG{p}{(}\PYG{n}{switches}\PYG{p}{,} \PYG{l+s}{"}\PYG{l+s}{LEDS}\PYG{l+s}{"}\PYG{p}{)}
\PYG{n}{serial\PYGZus{}out} \PYG{o}{=} \PYG{n}{SerialOut}\PYG{p}{(}\PYG{l+s}{"}\PYG{l+s}{TX}\PYG{l+s}{"}\PYG{p}{,} \PYG{n}{serial\PYGZus{}in}\PYG{p}{)}

\PYG{c}{\#We need to tell the *Chip* that *leds* and *serial\PYGZus{}out* are part of}
\PYG{c}{\#the device. The *Chip* can work out for itself that *switches* and}
\PYG{c}{\#*serial\PYGZus{}in* are part of the device.}

\PYG{n}{s} \PYG{o}{=} \PYG{n}{Chip}\PYG{p}{(}
    \PYG{n}{leds}\PYG{p}{,}
    \PYG{n}{serial\PYGZus{}out}\PYG{p}{,}
\PYG{p}{)}

\PYG{n}{s}\PYG{o}{.}\PYG{n}{write\PYGZus{}code}\PYG{p}{(}\PYG{n}{plugin}\PYG{p}{)}
\end{Verbatim}
\end{classdesc}


\section{Process}
\index{chips.process (module)}
\hypertarget{module-chips.process}{}
\declaremodule[chips.process]{}{chips.process}
\modulesynopsis{}
\emph{Processes} are used to define the programs that will be executed in the target
\emph{Chip}.  Each \emph{Process} contains a single program made up of instructions. When
a \emph{Chip} is simulated, or run in real hardware, the program within each process
will be run concurrently.


\subsection{Process Inputs}

Any \emph{Stream} may be used as the input to a \emph{Process}. Only one process may read
from any particular stream.  A \emph{Process} may read from a \emph{Stream} using the
\emph{read} method. The \emph{read} method accepts a \emph{Variable} as its argument. A \emph{read}
from a \emph{Stream} will stall execution of the \emph{Process} until data is available.
Similarly, the stream will be stalled, until data is read from it. This
provides a handy way to synchronise processes together, and simplifies the
design of concurrent systems.

Example:

\begin{Verbatim}[commandchars=\\\{\}]
\PYG{c}{\#sending process}
\PYG{n}{theoutput} \PYG{o}{=} \PYG{n}{Output}\PYG{p}{(}\PYG{p}{)}
\PYG{n}{count} \PYG{o}{=} \PYG{n}{Variable}\PYG{p}{(}\PYG{l+m+mi}{0}\PYG{p}{)}
\PYG{n}{Process}\PYG{p}{(}\PYG{l+m+mi}{16}\PYG{p}{,}
    \PYG{c}{\#wait for 1 second}
    \PYG{n}{count}\PYG{o}{.}\PYG{n}{set}\PYG{p}{(}\PYG{l+m+mi}{1000}\PYG{p}{)}\PYG{p}{,}
    \PYG{n}{While}\PYG{p}{(}\PYG{n}{count}\PYG{p}{,} 
        \PYG{n}{count}\PYG{o}{.}\PYG{n}{set}\PYG{p}{(}\PYG{n}{count}\PYG{o}{-}\PYG{l+m+mi}{1}\PYG{p}{)}\PYG{p}{,}
        \PYG{n}{WaitUs}\PYG{p}{(}\PYG{p}{)}
    \PYG{p}{)}\PYG{p}{,}
    \PYG{c}{\#send some data}
    \PYG{n}{theoutput}\PYG{o}{.}\PYG{n}{write}\PYG{p}{(}\PYG{l+m+mi}{123}\PYG{p}{)}\PYG{p}{,}
\PYG{p}{)}

\PYG{c}{\#receiving process}
\PYG{n}{target\PYGZus{}variable} \PYG{o}{=} \PYG{n}{Variable}\PYG{p}{(}\PYG{l+m+mi}{100}\PYG{p}{)}
\PYG{n}{Process}\PYG{p}{(}\PYG{l+m+mi}{16}\PYG{p}{,}
    \PYG{c}{\#This instruction will stall the process until data is available}
    \PYG{n}{theOutput}\PYG{o}{.}\PYG{n}{read}\PYG{p}{(}\PYG{n}{target\PYGZus{}variable}\PYG{p}{)}\PYG{p}{,}
    \PYG{c}{\#This instruction will not be run for 1 second}
    \PYG{c}{\#..}
\PYG{p}{)}
\end{Verbatim}


\subsection{Process Outputs}

An \emph{Output} is a special \emph{Stream} that can be written to by a \emph{Process}. Only one
\emph{Process} may write to any particular stream. Like any other \emph{Stream}, an
\emph{Output} may be:
\begin{itemize}
\item {} 
Read by a \emph{Process}.

\item {} 
Consumed by a \emph{Sink}.

\item {} 
Modified to form another \emph{Stream}.

\end{itemize}

A \emph{Process} may write to an \emph{Output} stream using the \emph{write} method. The
\emph{write} method accepts an expression as its argument. A \emph{write} to an output
will stall the process until the receiver is ready to receive data.

Example:

\begin{Verbatim}[commandchars=\\\{\}]
\PYG{c}{\#sending process}
\PYG{n}{theoutput} \PYG{o}{=} \PYG{n}{Output}\PYG{p}{(}\PYG{p}{)}
\PYG{n}{Process}\PYG{p}{(}\PYG{l+m+mi}{16}\PYG{p}{,}
    \PYG{c}{\#This instruction will stall the process until data is available}
    \PYG{n}{theoutput}\PYG{o}{.}\PYG{n}{write}\PYG{p}{(}\PYG{l+m+mi}{123}\PYG{p}{)}\PYG{p}{,}
    \PYG{c}{\#This instruction will not be run for 1 second}
    \PYG{c}{\#..}
\PYG{p}{)}

\PYG{c}{\#receiving process}
\PYG{n}{target\PYGZus{}variable} \PYG{o}{=} \PYG{n}{Variable}\PYG{p}{(}\PYG{l+m+mi}{0}\PYG{p}{)}
\PYG{n}{count} \PYG{o}{=} \PYG{n}{Variable}\PYG{p}{(}\PYG{l+m+mi}{0}\PYG{p}{)}
\PYG{n}{Process}\PYG{p}{(}\PYG{l+m+mi}{16}\PYG{p}{,}
    \PYG{c}{\#wait for 1 second}
    \PYG{n}{count}\PYG{o}{.}\PYG{n}{set}\PYG{p}{(}\PYG{l+m+mi}{1000}\PYG{p}{)}\PYG{p}{,}
    \PYG{n}{While}\PYG{p}{(}\PYG{n}{count}\PYG{p}{,} 
        \PYG{n}{count}\PYG{o}{.}\PYG{n}{set}\PYG{p}{(}\PYG{n}{count}\PYG{o}{-}\PYG{l+m+mi}{1}\PYG{p}{)}\PYG{p}{,}
        \PYG{n}{WaitUs}\PYG{p}{(}\PYG{p}{)}\PYG{p}{,}
    \PYG{p}{)}\PYG{p}{,}
    \PYG{c}{\#get some data}
    \PYG{n}{theoutput}\PYG{o}{.}\PYG{n}{read}\PYG{p}{(}\PYG{n}{target\PYGZus{}variable}\PYG{p}{)}\PYG{p}{,}
\PYG{p}{)}
\end{Verbatim}


\subsection{Variables}

Data is stored and manipulated within a process using \emph{Variables}. A \emph{Variable}
may only be accessed by one process.  When a \emph{Variable} an initial value must
be supplied. A variable will be reset to its initial value before any process
instructions are executed.  A \emph{Variable} may be assigned a value using the
\emph{set} method. The {\color{red}\bfseries{}*}set method accepts an expression as its argument.

It is important to understand that a \emph{Variable} object created like this:

\begin{Verbatim}[commandchars=\\\{\}]
\PYG{n}{a} \PYG{o}{=} \PYG{n}{Variable}\PYG{p}{(}\PYG{l+m+mi}{12}\PYG{p}{)}
\end{Verbatim}

is very different from a normal Python variable created like this:

\begin{Verbatim}[commandchars=\\\{\}]
\PYG{n}{a} \PYG{o}{=} \PYG{l+m+mi}{12}
\end{Verbatim}

The key is to understand that a \emph{Variable} will exist in the target \emph{Chip}, and
may be assigned and referenced as the \emph{Process} executes. A Python variable can
exist only in the Python environment, and not in a \emph{Chip}. While a Python
variable may be converted into a \emph{Constant} in the target \emph{Chip}, a \emph{Process}
has no way to change its value when it executes.


\subsection{Constants}

Like a \emph{Variable}, a constant must be supplied with an initial value when it
is created. Unlike a \emph{Variable}, the value of a \emph{Constant} can never be
changed.


\subsection{Expressions}

\emph{Variables} and \emph{Constants} are the most basic form of expressions. More
complex expressions can be formed by combining \emph{Constants}, \emph{Variables} and
other expressions using following operators:

\begin{Verbatim}[commandchars=@\[\]]
+, -, *, \/, @%, @&, @textbar[], @textasciicircum[], @textless[]@textless[], @textgreater[]@textgreater[], ==, !=, @textless[], @textless[]=, @textgreater[], @textgreater[]=
\end{Verbatim}

If one of the operands of a binary operator is not an expression, the Chips
library will attempt to convert this operand into an integer. If the conversion
is successful, a \emph{Constant} object will be created using the integer value.
The \emph{Constant} object will be used in place of the non-expression operand. This
allows constructs such as \code{a = 47+Constant(10)} to be used as a shorthand for
\code{a = Constant(47)+Constant(10)} or \code{count.set(Constant(15)+3*2} to be used
as a shorthand for \code{count.set(Constant(15)+Constant(6)}.  Of course \code{a=1+1}
still yields the integer 2 rather than an expression.

An expression within a process will always inherit the data width in bits of
the \emph{Process} in which it is evaluated. A \emph{Stream} expression such as
\code{Repeater(255) + 1} will automatically yield a 10-bit \emph{Stream} so that the
value 256 can be represented. A similar expression Constant(255)+1 will give an
9-bit result in a 9-bit process yielding the value -1. If the same expression
is evaluated in a 10-bit process, the result will be 256.


\subsection{Operator Precedence}

The operator precedence is inherited from the Python language. The following
table summarizes the operator precedences, from lowest precedence (least
binding) to highest precedence (most binding). Operators in the same row have
the same precedence.

\begin{tabulary}{\textwidth}{|L|L|}
\hline
\textbf{
Operator
} & \textbf{
Description
}\\
\hline

==, !=, \textless{}, \textless{}=, \textgreater{}, \textgreater{}=
 & 
Comparisons
\\
{\raggedright{}~}
 & 
Bitwise OR
\\

\textasciicircum{}
 & 
Bitwise XOR
\\

\&
 & 
Bitwise AND
\\

\textless{}\textless{}, \textgreater{}\textgreater{}
 & 
Shifts
\\

+, -
 & 
Addition and subtraction
\\

{\color{red}\bfseries{}*}, //, \%
 & 
multiplication, division and modulo
\\
\hline
\end{tabulary}

\index{Process (class in chips)}

\hypertarget{chips.Process}{}\begin{classdesc}{Process}{bits, *instructions}\end{classdesc}
\index{Variable (class in chips)}

\hypertarget{chips.Variable}{}\begin{classdesc}{Variable}{initial}\end{classdesc}
\index{VariableArray (class in chips)}

\hypertarget{chips.VariableArray}{}\begin{classdesc}{VariableArray}{size}\end{classdesc}


\section{Streams}
\index{chips.streams (module)}
\hypertarget{module-chips.streams}{}
\declaremodule[chips.streams]{}{chips.streams}
\modulesynopsis{}
Streams are a fundamental component of the \emph{Chips} library.
\begin{description}
\item[{A stream is used to represent a flow of data. A stream can act as a:}] \leavevmode\begin{itemize}
\item {} 
An input to a \emph{Chip} such as an \emph{InPort} or a \emph{SerialIn}.

\item {} 
A source of data in its own right such as a \emph{Repeater} or a \emph{Counter}.

\item {} 
A means of performing some operation on a stream of data to form 
another stream such as a \emph{Printer} or a \emph{Lookup}.

\item {} 
A means of transferring data from one process to another, an \emph{Output}.

\end{itemize}

\end{description}


\subsection{Stream Expressions}

A Stream Expression can be formed by combining Streams or Stream Expressions
with the following operators:

\begin{Verbatim}[commandchars=@\[\]]
+, -, *, \/, @%, @&, @textbar[], @textasciicircum[], @textless[]@textless[], @textgreater[]@textgreater[], ==, !=, @textless[], @textless[]=, @textgreater[], @textgreater[]=
\end{Verbatim}

Each data item in the resulting Stream Expression will be evaluated by removing
a data item from each of the operand streams, and applying the operator
function to these data items.

Generally speaking a Stream Expression will have enough bits to contain any
possible result without any arithmetic overflow. The one exception to this is
the left shift operator where the result is always truncated to the size of the
left hand operand. Stream expressions may be explicitly truncated or sign
extended using the \emph{Resizer}.

If one of the operands of a binary operator is not a Stream, Python Streams
will attempt to convert this operand into an integer. If the conversion is
successful, a \emph{Repeater} stream will be created using the integer value. The
repeater stream will be used in place of the non-stream operand. This allows
constructs such as \code{a = 47+InPort(12, 8)} to be used as a shorthand for \code{a =
Repeater(47)+InPort("in", 8)} or \code{count = Counter(1, 10, 1)+3*2} to be used as
a shorthand for \code{count = Counter(1, 10, 1)+Repeater(5)}.  Of course \code{a=1+1}
still yields the integer 2 rather than a stream.

The operators provided in the Python Streams library are summarised in the
table below. The bit width field specifies how many bits are used for the
result based on the number of bits in the left and right hand operands.

\begin{tabulary}{\textwidth}{|L|L|L|}
\hline
\textbf{
Operator
} & \textbf{
Function
} & \textbf{
Data Width (bits)
}\\
\hline

+
 & 
Signed Add
 & 
max(left, right) + 1
\\

-
 & 
Signed Subtract
 & 
max(left, right) + 1
\\

*
 & 
Signed Multiply
 & 
left + right
\\

//
 & 
Signed Floor Division
 & 
max(left, right) + 1
\\

\%
 & 
Signed Modulo
 & 
max(left, right)
\\

\&
 & 
Bitwise AND
 & 
max(left, right)
\\

\textbar{}
 & 
Bitwise OR
 & 
max(left, right)
\\

\textasciicircum{}
 & 
Bitwise XOR
 & 
max(left, right)
\\

\textless{}\textless{}
 & 
Arithmetic Left Shift
 & 
left
\\

\textgreater{}\textgreater{}
 & 
Arithmetic Right Shift
 & 
left
\\

==
 & 
Equality Comparison
 & 
1
\\

!=
 & 
Inequality Comparison
 & 
1
\\

\textless{}
 & 
Signed Less Than
Comparison
 & 
1
\\

\textless{}=
 & 
Signed Less Than or Equal
Comparison
 & 
1
\\

\textgreater{}
 & 
Signed Greater Than
Comparison
 & 
1
\\

\textgreater{}=
 & 
Signed Greater Than
Comparison
 & 
1
\\
\hline
\end{tabulary}



\subsection{Operator Precedence}

The operator precedence is inherited from the python language. The following
table summarizes the operator precedences, from lowest precedence (least
binding) to highest precedence (most binding). Operators in the same row have
the same precedence.

\begin{tabulary}{\textwidth}{|L|L|}
\hline
\textbf{
Operator
} & \textbf{
Description
}\\
\hline

==, !=, \textless{}, \textless{}=, \textgreater{}, \textgreater{}=
 & 
Comparisons
\\
{\raggedright{}~}
 & 
Bitwise OR
\\

\textasciicircum{}
 & 
Bitwise XOR
\\

\&
 & 
Bitwise AND
\\

\textless{}\textless{}, \textgreater{}\textgreater{}
 & 
Shifts
\\

+, -
 & 
Addition and subtraction
\\

{\color{red}\bfseries{}*}, //, \%
 & 
multiplication, division and modulo
\\
\hline
\end{tabulary}



\subsection{Streams Reference}
\index{Array (class in chips)}

\hypertarget{chips.Array}{}\begin{classdesc}{Array}{address\_in, data\_in, address\_out, depth}
An \emph{Array} is a stream yields values from a writeable lookup table.

Like a \emph{Lookup}, an \emph{Array} looks up each data item in the \emph{address\_in}
stream, and yields the value in the lookup table. In an \emph{Array}, the lookup
table is set up dynamically using data items from the \emph{address\_in} and
\emph{data\_in} streams. An \emph{Array} is equivalent to a Random Access Memory (RAM)
with independent read, and write ports.

A \emph{Lookup} accepts \emph{address\_in}, \emph{data\_in} and \emph{address\_out} arguments as
source streams. The \emph{depth} argument specifies the size of the lookup table.

Example:

\begin{Verbatim}[commandchars=\\\{\}]
\PYG{k}{def} \PYG{n+nf}{video\PYGZus{}raster\PYGZus{}stream}\PYG{p}{(}\PYG{n}{width}\PYG{p}{,} \PYG{n}{height}\PYG{p}{,} \PYG{n}{row\PYGZus{}stream}\PYG{p}{,} \PYG{n}{col\PYGZus{}stream}\PYG{p}{,} 
                        \PYG{n}{pixel\PYGZus{}intensity}\PYG{p}{)}\PYG{p}{:}

    \PYG{n}{pixel\PYGZus{}clock} \PYG{o}{=} \PYG{n}{Counter}\PYG{p}{(}\PYG{l+m+mi}{0}\PYG{p}{,} \PYG{n}{width}\PYG{o}{*}\PYG{n}{height}\PYG{p}{,} \PYG{l+m+mi}{1}\PYG{p}{)}
    \PYG{n}{red\PYGZus{}intensity}\PYG{p}{,} \PYG{n}{green\PYGZus{}intensity}\PYG{p}{,} \PYG{n}{blue\PYGZus{}intensity} \PYG{o}{=} \PYG{n}{pixel\PYGZus{}intensity}
     
    \PYG{n}{red} \PYG{o}{=} \PYG{n}{Array}\PYG{p}{(}
        \PYG{n}{address\PYGZus{}in} \PYG{o}{=} \PYG{n}{row\PYGZus{}stream} \PYG{o}{*} \PYG{n}{width\PYGZus{}stream} \PYG{o}{+} \PYG{n}{col\PYGZus{}stream}\PYG{p}{,}
        \PYG{n}{data\PYGZus{}in} \PYG{o}{=} \PYG{n}{red\PYGZus{}intensity}\PYG{p}{,}
        \PYG{n}{address\PYGZus{}out} \PYG{o}{=} \PYG{n}{pixel\PYGZus{}clock}\PYG{p}{,}
        \PYG{n}{depth} \PYG{o}{=} \PYG{n}{width} \PYG{o}{*} \PYG{n}{height}\PYG{p}{,}
    \PYG{p}{)}

    \PYG{n}{green} \PYG{o}{=} \PYG{n}{Array}\PYG{p}{(}
        \PYG{n}{address\PYGZus{}in} \PYG{o}{=} \PYG{n}{row\PYGZus{}stream} \PYG{o}{*} \PYG{n}{width\PYGZus{}stream} \PYG{o}{+} \PYG{n}{col\PYGZus{}stream}\PYG{p}{,}
        \PYG{n}{data\PYGZus{}in} \PYG{o}{=} \PYG{n}{green\PYGZus{}intensity}\PYG{p}{,}
        \PYG{n}{address\PYGZus{}out} \PYG{o}{=} \PYG{n}{pixel\PYGZus{}clock}\PYG{p}{,}
        \PYG{n}{depth} \PYG{o}{=} \PYG{n}{width} \PYG{o}{*} \PYG{n}{height}\PYG{p}{,}
    \PYG{p}{)}

    \PYG{n}{blue} \PYG{o}{=} \PYG{n}{Array}\PYG{p}{(}
        \PYG{n}{address\PYGZus{}in} \PYG{o}{=} \PYG{n}{row\PYGZus{}stream} \PYG{o}{*} \PYG{n}{width\PYGZus{}stream} \PYG{o}{+} \PYG{n}{col\PYGZus{}stream}\PYG{p}{,}
        \PYG{n}{data\PYGZus{}in} \PYG{o}{=} \PYG{n}{blue\PYGZus{}intensity}\PYG{p}{,}
        \PYG{n}{address\PYGZus{}out} \PYG{o}{=} \PYG{n}{pixel\PYGZus{}clock}\PYG{p}{,}
        \PYG{n}{depth} \PYG{o}{=} \PYG{n}{width} \PYG{o}{*} \PYG{n}{height}\PYG{p}{,}
    \PYG{p}{)}

    \PYG{k}{return} \PYG{n}{red}\PYG{p}{,} \PYG{n}{green}\PYG{p}{,} \PYG{n}{blue}
\end{Verbatim}
\end{classdesc}
\index{Counter (class in chips)}

\hypertarget{chips.Counter}{}\begin{classdesc}{Counter}{start, stop, step}
A Stream which yields numbers from \emph{start} to \emph{stop} in \emph{step} increments.

A \emph{Counter} is a versatile, and commonly used construct in device design,
they can be used to number samples, index memories and so on.

Example:

\begin{Verbatim}[commandchars=\\\{\}]
\PYG{n}{Counter}\PYG{p}{(}\PYG{l+m+mi}{0}\PYG{p}{,} \PYG{l+m+mi}{10}\PYG{p}{,} \PYG{l+m+mi}{2}\PYG{p}{)} \PYG{c}{\# --\textgreater{} 0 2 4 6 8 10 0 \PYGZbs{}..}

\PYG{n}{Counter}\PYG{p}{(}\PYG{l+m+mi}{10}\PYG{p}{,} \PYG{l+m+mi}{0}\PYG{p}{,} \PYG{o}{-}\PYG{l+m+mi}{2}\PYG{p}{)} \PYG{c}{\# --\textgreater{} 10 8 7 6 4 2 0 10 \PYGZbs{}..}
\end{Verbatim}
\end{classdesc}
\index{Decoupler (class in chips)}

\hypertarget{chips.Decoupler}{}\begin{classdesc}{Decoupler}{source}
A \emph{Decoupler} removes stream handshaking.

Usually, data is transfered though streams using blocking transfers. When a
process writes to a stream, execution will be halted until the receiving
process reads the data. While this behaviour greatly simplifies the design
of parallel processes, sometimes Non-blocking transfers are needed. When a
data item is written to a \emph{Decoupler}, it is stored. When a \emph{Decoupler} is
read from, the value of the last stored value is yielded. Neither the
sending or the receiving process ever blocks. This also means that the
number of data items written into the \emph{Decoupler} and the number read out
do not have to be the same.

A \emph{Decoupler} accepts only one argument, the source stream.

Example:

\begin{Verbatim}[commandchars=\\\{\}]
\PYG{k}{def} \PYG{n+nf}{time\PYGZus{}stamp\PYGZus{}data}\PYG{p}{(}\PYG{n}{data\PYGZus{}stream}\PYG{p}{)}\PYG{p}{:}

    \PYG{n}{us\PYGZus{}time} \PYG{o}{=} \PYG{n}{Output}\PYG{p}{(}\PYG{p}{)}
    \PYG{n}{time} \PYG{o}{=} \PYG{n}{Variable}\PYG{p}{(}\PYG{l+m+mi}{0}\PYG{p}{)}
    \PYG{n}{Process}\PYG{p}{(}\PYG{l+m+mi}{8}\PYG{p}{,}
        \PYG{n}{Loop}\PYG{p}{(}
            \PYG{n}{WaitUs}\PYG{p}{,}
            \PYG{n}{time}\PYG{o}{.}\PYG{n}{set}\PYG{p}{(}\PYG{n}{time} \PYG{o}{+} \PYG{l+m+mi}{1}\PYG{p}{)}\PYG{p}{,}
            \PYG{n}{us\PYGZus{}time}\PYG{o}{.}\PYG{n}{write}\PYG{p}{(}\PYG{n}{time}\PYG{p}{)}\PYG{p}{,}
        \PYG{p}{)}\PYG{p}{,}
    \PYG{p}{)}

    \PYG{n}{output\PYGZus{}stream} \PYG{o}{=} \PYG{n}{Output}\PYG{p}{(}\PYG{p}{)}
    \PYG{n}{temp} \PYG{o}{=} \PYG{n}{Variable}\PYG{p}{(}\PYG{l+m+mi}{0}\PYG{p}{)}
    \PYG{n}{Process}\PYG{p}{(}\PYG{l+m+mi}{8}\PYG{p}{,}
        \PYG{n}{Loop}\PYG{p}{(}
            \PYG{n}{data\PYGZus{}stream}\PYG{o}{.}\PYG{n}{read}\PYG{p}{(}\PYG{n}{temp}\PYG{p}{)}\PYG{p}{,}
            \PYG{n}{output\PYGZus{}stream}\PYG{o}{.}\PYG{n}{write}\PYG{p}{(}\PYG{n}{temp}\PYG{p}{)}\PYG{p}{,}
            \PYG{n}{us\PYGZus{}time}\PYG{o}{.}\PYG{n}{read}\PYG{p}{(}\PYG{n}{temp}\PYG{p}{)}\PYG{p}{,}
            \PYG{n}{output\PYGZus{}stream}\PYG{o}{.}\PYG{n}{write}\PYG{p}{(}\PYG{n}{temp}\PYG{p}{)}\PYG{p}{,}
        \PYG{p}{)}\PYG{p}{,}
    \PYG{p}{)}

    \PYG{k}{return} \PYG{n}{output\PYGZus{}stream}
\end{Verbatim}
\end{classdesc}
\index{Resizer (class in chips)}

\hypertarget{chips.Resizer}{}\begin{classdesc}{Resizer}{source, bits}
A \emph{Resizer} changes the width, in bits, of the source stream.

The \emph{Resizer} takes two arguments, the source stream, and the \emph{width} in
bits. The \emph{Resizer} will truncate data if it is reducing the width, ans
sign extend if it is increasing the width.

Example:

\begin{Verbatim}[commandchars=@\[\]]
a = InPort(name="din", bits=8) a has a width of 8 bits
b = Inport + 1 @#b has a width of 9 bits
c = Resizer(b, 8) @#c is truncated to 8 bits
Chip(OutPort(name="dout"))
\end{Verbatim}
\end{classdesc}
\index{Lookup (class in chips)}

\hypertarget{chips.Lookup}{}\begin{classdesc}{Lookup}{source, *args}
A \emph{Lookup} is a stream yields values from a read-only look up table.

For each data item in the source stream, a \emph{Lookup} will yield the
addressed value in the lookup table. A \emph{Lookup} is basically a Read Only
Memory(ROM) with the source stream forming the address, and the \emph{Lookup}
itself forming the data output.

Example:

\begin{Verbatim}[commandchars=\\\{\}]
\PYG{k}{def} \PYG{n+nf}{binary\PYGZus{}2\PYGZus{}gray}\PYG{p}{(}\PYG{n}{input\PYGZus{}stream}\PYG{p}{)}\PYG{p}{:} 
    \PYG{k}{return} \PYG{n}{Lookup}\PYG{p}{(}\PYG{n}{input\PYGZus{}stream}\PYG{p}{,} \PYG{l+m+mi}{0}\PYG{p}{,} \PYG{l+m+mi}{1}\PYG{p}{,} \PYG{l+m+mi}{3}\PYG{p}{,} \PYG{l+m+mi}{2}\PYG{p}{,} \PYG{l+m+mi}{6}\PYG{p}{,} \PYG{l+m+mi}{7}\PYG{p}{,} \PYG{l+m+mi}{5}\PYG{p}{,} \PYG{l+m+mi}{4}\PYG{p}{)}
\end{Verbatim}

The first argument to a \emph{Lookup} is the source stream, all additional
arguments form the lookup table. If you want to use a Python sequence
object such as a tuple or a list to form the lookup table use the following
syntax:

\begin{Verbatim}[commandchars=\\\{\}]
\PYG{n}{my\PYGZus{}list} \PYG{o}{=} \PYG{p}{[}\PYG{l+m+mi}{0}\PYG{p}{,} \PYG{l+m+mi}{1}\PYG{p}{,} \PYG{l+m+mi}{3}\PYG{p}{,} \PYG{l+m+mi}{2}\PYG{p}{,} \PYG{l+m+mi}{6}\PYG{p}{,} \PYG{l+m+mi}{7}\PYG{p}{,} \PYG{l+m+mi}{5}\PYG{p}{,} \PYG{l+m+mi}{4}\PYG{p}{]} 
\PYG{n}{my\PYGZus{}sequence} \PYG{o}{=} \PYG{n}{Sequence}\PYG{p}{(}\PYG{n}{Counter}\PYG{p}{(}\PYG{l+m+mi}{0}\PYG{p}{,} \PYG{l+m+mi}{7}\PYG{p}{,} \PYG{l+m+mi}{1}\PYG{p}{)}\PYG{p}{,} \PYG{o}{*}\PYG{n}{my\PYGZus{}list}\PYG{p}{)}
\end{Verbatim}
\end{classdesc}
\index{Fifo (class in chips)}

\hypertarget{chips.Fifo}{}\begin{classdesc}{Fifo}{data\_in, depth}
A \emph{Fifo} stores a buffer of data items.

A \emph{Fifo} contains a fixed size buffer of objects obtained from the source
stream. A \emph{Fifo} yields the data items in the same order in which they were
stored.

The first argument to a \emph{Fifo}, is the source stream, the \emph{depth} argument
determines the size of the Fifo buffer.

Example:

\begin{Verbatim}[commandchars=\\\{\}]
\PYG{k}{def} \PYG{n+nf}{digital\PYGZus{}oscilloscope}\PYG{p}{(}\PYG{n}{ADC\PYGZus{}stream}\PYG{p}{,} \PYG{n}{trigger\PYGZus{}level}\PYG{p}{)}\PYG{p}{:} 
    \PYG{n}{temp} \PYG{o}{=} \PYG{n}{Variable}\PYG{p}{(}\PYG{l+m+mi}{0}\PYG{p}{)}
    \PYG{n}{count} \PYG{o}{=} \PYG{n}{Variable}\PYG{p}{(}\PYG{l+m+mi}{0}\PYG{p}{)}

    \PYG{n}{Process}\PYG{p}{(}\PYG{l+m+mi}{16}\PYG{p}{,}
        \PYG{n}{Loop}\PYG{p}{(}
            \PYG{n}{ADC\PYGZus{}stream}\PYG{o}{.}\PYG{n}{read}\PYG{p}{(}\PYG{n}{temp}\PYG{p}{)}\PYG{p}{,}
            \PYG{n}{If}\PYG{p}{(}\PYG{n}{temp} \PYG{o}{\textgreater{}} \PYG{n}{trigger\PYGZus{}level}\PYG{p}{,}
                \PYG{n}{count}\PYG{o}{.}\PYG{n}{set}\PYG{p}{(}\PYG{n}{buffer\PYGZus{}depth}\PYG{p}{)}\PYG{p}{,}
                \PYG{n}{While}\PYG{p}{(}\PYG{n}{count}\PYG{p}{,}
                    \PYG{n}{ADC\PYGZus{}stream}\PYG{o}{.}\PYG{n}{read}\PYG{p}{(}\PYG{n}{temp}\PYG{p}{)}\PYG{p}{,}
                    \PYG{n+nb}{buffer}\PYG{o}{.}\PYG{n}{write}\PYG{p}{(}\PYG{n}{temp}\PYG{p}{)}\PYG{p}{,}
                    \PYG{n}{count}\PYG{o}{.}\PYG{n}{set}\PYG{p}{(}\PYG{n}{count}\PYG{o}{-}\PYG{l+m+mi}{1}\PYG{p}{)}\PYG{p}{,}
                \PYG{p}{)}\PYG{p}{,}
            \PYG{p}{)}\PYG{p}{,}
        \PYG{p}{)}\PYG{p}{,}
    \PYG{p}{)}
            
    \PYG{k}{return} \PYG{n}{SerialOut}\PYG{p}{(}\PYG{n}{Printer}\PYG{p}{(}\PYG{n}{Fifo}\PYG{p}{(}\PYG{n+nb}{buffer}\PYG{p}{,} \PYG{n}{buffer\PYGZus{}depth}\PYG{p}{)}\PYG{p}{)}\PYG{p}{)}
\end{Verbatim}
\end{classdesc}
\index{Repeater (class in chips)}

\hypertarget{chips.Repeater}{}\begin{classdesc}{Repeater}{value}
A stream which repeatedly yields the specified \emph{value}.

The \emph{Repeater} stream is one of the most fundamental streams available.

The width of the stream in bits is calculated automatically. The smallest
number of bits that can represent \emph{value} in twos-complement format will be
used.

Examples:

\begin{Verbatim}[commandchars=\\\{\}]
\PYG{n}{Repeater}\PYG{p}{(}\PYG{l+m+mi}{5}\PYG{p}{)} \PYG{c}{\#--\textgreater{} 5 5 5 5 \PYGZbs{}..}
\PYG{c}{\#creates a 4 bit stream.}

\PYG{n}{Repeater}\PYG{p}{(}\PYG{l+m+mi}{10}\PYG{p}{)} \PYG{c}{\#--\textgreater{} 10 10 10 10 \PYGZbs{}..}
\PYG{c}{\#creates a 5 bit stream.}

\PYG{n}{Repeater}\PYG{p}{(}\PYG{l+m+mi}{5}\PYG{p}{)}\PYG{o}{*}\PYG{l+m+mi}{2} \PYG{c}{\#--\textgreater{} 10 10 10 10 \PYGZbs{}..}
\PYG{c}{\#This is shorthand for: Repeater(5)*Repeater(2)}
\end{Verbatim}
\end{classdesc}
\index{Sequence (class in chips)}

\hypertarget{chips.Sequence}{}\begin{classdesc}{Sequence}{}\end{classdesc}
\index{Stimulus (class in chips)}

\hypertarget{chips.Stimulus}{}\begin{classdesc}{Stimulus}{bits}
A Stream that allows a Python iterable to be used as a stream.

A Stimulus stream allows a transparent method to pass data from the Python
environment into the simulation environment. The sequence object is set at
run time using the set\_simulation\_data() method. The sequence object can be
any iterable Python sequence such as a list, tuple, or even a generator.

Example:

\begin{Verbatim}[commandchars=\\\{\}]
\PYG{k+kn}{import} \PYG{n+nn}{PIL}

\PYG{n}{picture} \PYG{o}{=} \PYG{n}{Stimulus}\PYG{p}{(}\PYG{p}{)}
\PYG{n}{s} \PYG{o}{=} \PYG{n}{Chip}\PYG{p}{(}\PYG{n}{Console}\PYG{p}{(}\PYG{n}{Printer}\PYG{p}{(}\PYG{n}{picture}\PYG{p}{)}\PYG{p}{)}\PYG{p}{)}

\PYG{n}{im} \PYG{o}{=} \PYG{n}{PIL}\PYG{o}{.}\PYG{n}{open}\PYG{p}{(}\PYG{l+s}{"}\PYG{l+s}{test.bmp}\PYG{l+s}{"}\PYG{p}{)}
\PYG{n}{image\PYGZus{}data} \PYG{o}{=} \PYG{n+nb}{list}\PYG{p}{(}\PYG{n}{im}\PYG{o}{.}\PYG{n}{getdata}\PYG{p}{(}\PYG{p}{)}\PYG{p}{)}
\PYG{n}{picture}\PYG{o}{.}\PYG{n}{set\PYGZus{}simulation\PYGZus{}data}\PYG{p}{(}\PYG{n}{image\PYGZus{}data}\PYG{p}{)}

\PYG{n}{picture}\PYG{o}{.}\PYG{n}{reset}\PYG{p}{(}\PYG{p}{)}
\PYG{n}{picture}\PYG{o}{.}\PYG{n}{execute}\PYG{p}{(}\PYG{l+m+mi}{1000}\PYG{p}{)}
\end{Verbatim}
\end{classdesc}
\index{InPort (class in chips)}

\hypertarget{chips.InPort}{}\begin{classdesc}{InPort}{name, bits}
A device input port stream.

An \emph{InPort} allows a port pins of the target device to be used as a data
stream.  There is no handshaking on the input port. The port pins are
sampled at the point when data is transfered by the stream.  When
implemented in VHDL, the \emph{InPort} provides double registers on the port
pins to synchronise data to the local clock domain.

Since it is not possible to determine the width of the stream in bits
automatically, this must be specified using the \emph{bits} argument.

The \emph{name} parameter allows a string to be associated with the input port.
In a VHDL implementation, \emph{name} will be used as the port name in the
top level entity.

Example:

\begin{Verbatim}[commandchars=\\\{\}]
\PYG{n}{dip\PYGZus{}switches} \PYG{o}{=} \PYG{n}{Inport}\PYG{p}{(}\PYG{l+s}{"}\PYG{l+s}{dip\PYGZus{}switches}\PYG{l+s}{"}\PYG{p}{,} \PYG{l+m+mi}{8}\PYG{p}{)} 
\PYG{n}{s} \PYG{o}{=} \PYG{n}{Chip}\PYG{p}{(}\PYG{n}{SerialOut}\PYG{p}{(}\PYG{n}{Printer}\PYG{p}{(}\PYG{n}{dip\PYGZus{}switches}\PYG{p}{)}\PYG{p}{)}\PYG{p}{)}
\end{Verbatim}
\end{classdesc}
\index{SerialIn (class in chips)}

\hypertarget{chips.SerialIn}{}\begin{classdesc}{SerialIn}{name='RX', clock\_rate=50000000, baud\_rate=115200}
A \emph{SerialIn} yields data from a serial UART port.

\emph{SerialIn} yields one data item from the serial input port for each
character read from the source stream.  The stream is always 8 bits wide.

A \emph{SerialIn} accepts an optional \emph{name} argument which is used as the name
for the serial RX line in generated VHDL. The clock rate of the target
device in MHz can be specified using the \emph{clock\_rate} argument. The baud
rate of the serial input can be specified using the \emph{baud\_rate} argument.

Example:

\begin{Verbatim}[commandchars=@\[\]]
@#echo typed characters
my@_chip = Chip(SerialOut(SerialIn())
\end{Verbatim}
\end{classdesc}
\index{Output (class in chips)}

\hypertarget{chips.Output}{}\begin{classdesc}{Output}{}
An \emph{Output} is a stream that can be written to by a process.

Any stream can be read from by a process. Only an \emph{Output} stream can be
written to by a process.  A process can be written to by using the \emph{read}
method. The read method accepts one argument, an expression to write.

Example:

\begin{Verbatim}[commandchars=\\\{\}]
\PYG{k}{def} \PYG{n+nf}{tee}\PYG{p}{(}\PYG{n}{input\PYGZus{}stream}\PYG{p}{)}\PYG{p}{:}
    \PYG{n}{output\PYGZus{}stream\PYGZus{}1} \PYG{o}{=} \PYG{n}{Output}\PYG{p}{(}\PYG{p}{)}
    \PYG{n}{output\PYGZus{}stream\PYGZus{}2} \PYG{o}{=} \PYG{n}{Output}\PYG{p}{(}\PYG{p}{)}
    \PYG{n}{temp} \PYG{o}{=} \PYG{n}{Variable}\PYG{p}{(}\PYG{l+m+mi}{0}\PYG{p}{)}
    \PYG{n}{Process}\PYG{p}{(}\PYG{n}{input\PYGZus{}stream}\PYG{o}{.}\PYG{n}{get\PYGZus{}bits}\PYG{p}{,}
        \PYG{n}{Loop}\PYG{p}{(}
            \PYG{n}{input\PYGZus{}stream}\PYG{o}{.}\PYG{n}{read}\PYG{p}{(}\PYG{n}{temp}\PYG{p}{)}\PYG{p}{,}
            \PYG{n}{output\PYGZus{}stream\PYGZus{}1}\PYG{o}{.}\PYG{n}{write}\PYG{p}{(}\PYG{n}{temp}\PYG{p}{)}\PYG{p}{,}
            \PYG{n}{output\PYGZus{}stream\PYGZus{}2}\PYG{o}{.}\PYG{n}{write}\PYG{p}{(}\PYG{n}{temp}\PYG{p}{)}\PYG{p}{,}
        \PYG{p}{)}
    \PYG{p}{)}
    \PYG{k}{return} \PYG{n}{input\PYGZus{}stream\PYGZus{}1}\PYG{p}{,} \PYG{n}{input\PYGZus{}stream\PYGZus{}2}
\end{Verbatim}
\end{classdesc}
\index{Printer (class in chips)}

\hypertarget{chips.Printer}{}\begin{classdesc}{Printer}{source}
A \emph{Printer} turns data into decimal ASCII characters.

Each each data item is turned into the ASCII representation of its decimal
value, terminated with a newline character. Each character then forms a
data item in the \emph{Printer} stream.

A \emph{Printer} accepts a single argument, the source stream. A \emph{Printer}
stream is always 8 bits wide.

Example:

\begin{Verbatim}[commandchars=\\\{\}]
\PYG{c}{\#print the numbers 0-10 to the console repeatedly}
\PYG{n}{Chip}\PYG{p}{(}
    \PYG{n}{Console}\PYG{p}{(}
        \PYG{n}{Printer}\PYG{p}{(}
            \PYG{n}{Counter}\PYG{p}{(}\PYG{l+m+mi}{0}\PYG{p}{,} \PYG{l+m+mi}{10}\PYG{p}{,} \PYG{l+m+mi}{1}\PYG{p}{)}\PYG{p}{,}
        \PYG{p}{)}\PYG{p}{,}
    \PYG{p}{)}\PYG{p}{,}
\PYG{p}{)}
\end{Verbatim}
\end{classdesc}
\index{HexPrinter (class in chips)}

\hypertarget{chips.HexPrinter}{}\begin{classdesc}{HexPrinter}{source}
A \emph{HexPrinter} turns data into hexadecimal ASCII characters.

Each each data item is turned into the ASCII representation of its
hexadecimal value, terminated with a newline character. Each character then
forms a data item in the \emph{HexPrinter} stream.

A \emph{HexPrinter} accepts a single argument, the source stream. A \emph{HexPrinter}
stream is always 8 bits wide.

Example:

\begin{Verbatim}[commandchars=\\\{\}]
\PYG{c}{\#print the numbers 0x0-0x10 to the console repeatedly}
\PYG{n}{Chip}\PYG{p}{(}
    \PYG{n}{Console}\PYG{p}{(}
        \PYG{n}{Printer}\PYG{p}{(}
            \PYG{n}{Counter}\PYG{p}{(}\PYG{l+m+mh}{0x0}\PYG{p}{,} \PYG{l+m+mh}{0x10}\PYG{p}{,} \PYG{l+m+mi}{1}\PYG{p}{)}\PYG{p}{,}
        \PYG{p}{)}\PYG{p}{,}
    \PYG{p}{)}\PYG{p}{,}
\PYG{p}{)}
\end{Verbatim}
\end{classdesc}
\index{Scanner (class in chips)}

\hypertarget{chips.Scanner}{}\begin{classdesc}{Scanner}{}\end{classdesc}


\section{Sinks}
\index{chips.sinks (module)}
\hypertarget{module-chips.sinks}{}
\declaremodule[chips.sinks]{}{chips.sinks}
\modulesynopsis{}
Sinks are a fundamental component of the \emph{Chips} library.
\begin{description}
\item[{A sink is used to terminate a stream. A sink may act as:}] \leavevmode\begin{itemize}
\item {} 
An output of a \emph{Chip} such as an \emph{OutPort} or \emph{SerialOut}.

\item {} 
A consumer of data in its own right such as an \emph{Asserter}.

\end{itemize}

\end{description}


\subsection{Sinks Reference}
\index{Response (class in chips)}

\hypertarget{chips.Response}{}\begin{classdesc}{Response}{a}
A \emph{Response} sink allows data to be transfered into Python.

As a simulation is run, the \emph{Response} sink accumulates data. After a
simulation is run, you can retrieve a python iterable using the
get\_simulation\_data method. Using a \emph{Response} sink allows you to
seamlessly integrate your \emph{Chips} simulation into a wider Python
simulation. This works for simulations using an external simulator as well,
in this case you also need to pass the code generation plugin to
get\_simulation\_data.

A \emph{Response} sink accepts a single stream argument as its source.

Example:

\begin{Verbatim}[commandchars=\\\{\}]
\PYG{k+kn}{import} \PYG{n+nn}{PIL}

\PYG{k}{def} \PYG{n+nf}{image\PYGZus{}processor}\PYG{p}{(}\PYG{p}{)}\PYG{p}{:}
    \PYG{c}{\#some image processing algorithm}
    \PYG{k}{pass}

\PYG{n}{response} \PYG{o}{=} \PYG{n}{Response}\PYG{p}{(}\PYG{n}{image\PYGZus{}processor}\PYG{p}{)}
\PYG{n}{chip} \PYG{o}{=} \PYG{n}{Chip}\PYG{p}{(}\PYG{n}{response}\PYG{p}{)}

\PYG{n}{chip}\PYG{o}{.}\PYG{n}{reset}\PYG{p}{(}\PYG{p}{)}
\PYG{n}{chip}\PYG{o}{.}\PYG{n}{execute}\PYG{p}{(}\PYG{l+m+mi}{10000}\PYG{p}{)}


\PYG{n}{image\PYGZus{}data} \PYG{o}{=} \PYG{n+nb}{list}\PYG{p}{(}\PYG{n}{response}\PYG{o}{.}\PYG{n}{get\PYGZus{}simulation\PYGZus{}data}\PYG{p}{(}\PYG{n}{plugin}\PYG{p}{)}\PYG{p}{)}
\PYG{n}{im} \PYG{o}{=} \PYG{n}{PIL}\PYG{o}{.}\PYG{n}{Image}\PYG{o}{.}\PYG{n}{new}\PYG{p}{(}\PYG{l+s}{"}\PYG{l+s}{L}\PYG{l+s}{"}\PYG{p}{,} \PYG{p}{(}\PYG{l+m+mi}{64}\PYG{p}{,} \PYG{l+m+mi}{64}\PYG{p}{)}\PYG{p}{)}
\PYG{n}{im}\PYG{o}{.}\PYG{n}{putdata}\PYG{p}{(}\PYG{n}{image\PYGZus{}data}\PYG{p}{)}
\PYG{n}{im}\PYG{o}{.}\PYG{n}{show}\PYG{p}{(}\PYG{p}{)}
\end{Verbatim}
\end{classdesc}
\index{OutPort (class in chips)}

\hypertarget{chips.OutPort}{}\begin{classdesc}{OutPort}{a, name}
An \emph{OutPort} sink outputs a stream of data to I/O port pins.

No handshaking is performed on the output port, data will appear at the
time when the source stream transfers data.

An output port take two arguments, the source stream \emph{a} and a string
\emph{name}. Name is used as the port name in generated VHDL.

Example:

\begin{Verbatim}[commandchars=\\\{\}]
\PYG{n}{dip\PYGZus{}switches} \PYG{o}{=} \PYG{n}{Inport}\PYG{p}{(}\PYG{l+s}{"}\PYG{l+s}{dip\PYGZus{}switches}\PYG{l+s}{"}\PYG{p}{,} \PYG{l+m+mi}{8}\PYG{p}{)} 
\PYG{n}{led\PYGZus{}array} \PYG{o}{=} \PYG{n}{OutPort}\PYG{p}{(}\PYG{n}{dip\PYGZus{}switched}\PYG{p}{,} \PYG{l+s}{"}\PYG{l+s}{led\PYGZus{}array}\PYG{l+s}{"}\PYG{p}{)}
\PYG{n}{s} \PYG{o}{=} \PYG{n}{Chip}\PYG{p}{(}\PYG{n}{led\PYGZus{}array}\PYG{p}{)}
\end{Verbatim}
\end{classdesc}
\index{SerialOut (class in chips)}

\hypertarget{chips.SerialOut}{}\begin{classdesc}{SerialOut}{a, name='TX', clock\_rate=50000000, baud\_rate=115200}
A \emph{SerialOut} outputs data to a serial UART port.

\emph{SerialOut} outputs one character to the serial output port for each item
of data in the source stream. At present only 8 data bits are supported, so
the source stream must be 8 bits wide. The source stream could be truncated
to 8 bits using a \emph{Resizer}, but it is usually more convenient to use a
\emph{Printer} as the source stream. The will allow a stream of any width to be
represented as a decimal string.

A SerialOut accepts a source stream argument \emph{a}. An optional \emph{name}
argument is used as the name for the serial TX line in generated VHDL. The
clock rate of the target device in MHz can be specified using the
\emph{clock\_rate} argument. The baud rate of the serial output can be specified
using the \emph{baud\_rate} argument.

Example:

\begin{Verbatim}[commandchars=\\\{\}]
\PYG{c}{\#convert string into a sequence of characters}
\PYG{n}{hello\PYGZus{}world} \PYG{o}{=} \PYG{n+nb}{tuple}\PYG{p}{(}\PYG{p}{(}\PYG{n+nb}{ord}\PYG{p}{(}\PYG{n}{i}\PYG{p}{)} \PYG{k}{for} \PYG{n}{i} \PYG{o+ow}{in} \PYG{l+s}{"}\PYG{l+s}{hello world}\PYG{l+s+se}{\PYGZbs{}n}\PYG{l+s}{"}\PYG{p}{)}\PYG{p}{)}

\PYG{n}{my\PYGZus{}chip} \PYG{o}{=} \PYG{n}{Chip}\PYG{p}{(}
    \PYG{n}{SerialOut}\PYG{p}{(}
        \PYG{n}{Sequence}\PYG{p}{(}\PYG{o}{*}\PYG{n}{hello\PYGZus{}world}\PYG{p}{)}\PYG{p}{,}
    \PYG{p}{)}
\PYG{p}{)}
\end{Verbatim}
\end{classdesc}
\index{Asserter (class in chips)}

\hypertarget{chips.Asserter}{}\begin{classdesc}{Asserter}{a}
An \emph{Asserter} causes an exception if any data in the source stream is zero.

An \emph{Asserter} is particularly useful in automated tests, as it causes a
simulation to fail is a condition is not met. In generated VHDL code, an
asserter is represented by a VHDL assert statement. In practice this means
that an \emph{Asserter} will function correctly in a VHDL simulation, but will
have no effect when synthesized.

The \emph{Asserter} sink accepts a source stream argument, \emph{a}.

Example:

\begin{Verbatim}[commandchars=\\\{\}]
\PYG{n}{a} \PYG{o}{=} \PYG{n}{Sequence}\PYG{p}{(}\PYG{l+m+mi}{1}\PYG{p}{,} \PYG{l+m+mi}{2}\PYG{p}{,} \PYG{l+m+mi}{3}\PYG{p}{,} \PYG{l+m+mi}{4}\PYG{p}{)}
\PYG{n}{Chip}\PYG{p}{(}\PYG{n}{Asserter}\PYG{p}{(}\PYG{p}{(}\PYG{n}{a}\PYG{o}{+}\PYG{l+m+mi}{1}\PYG{p}{)} \PYG{o}{==} \PYG{n}{Sequence}\PYG{p}{(}\PYG{l+m+mi}{2}\PYG{p}{,} \PYG{l+m+mi}{3}\PYG{p}{,} \PYG{l+m+mi}{4}\PYG{p}{,} \PYG{l+m+mi}{5}\PYG{p}{)}\PYG{p}{)}\PYG{p}{)}
\end{Verbatim}

Look at the Chips test suite for more examples of the Asserter being used
for automated testing.
\end{classdesc}
\index{Console (class in chips)}

\hypertarget{chips.Console}{}\begin{classdesc}{Console}{a}
A \emph{Console} outputs data to the simulation console.

\emph{Console} stores characters for output to the console in a buffer. When an
end of line character is seen, the buffer is written to the console.  A
\emph{Console} interprets a stream of numbers as ASCII characters. The source
stream must be 8 bits wide. The source stream could be truncated to 8 bits
using a \emph{Resizer}, but it is usually more convenient to use a \emph{Printer} as
the source stream. The will allow a stream of any width to be represented
as a decimal string.

A \emph{Console} accepts a source stream argument \emph{a}.

Example:

\begin{Verbatim}[commandchars=\\\{\}]
\PYG{c}{\#convert string into a sequence of characters}
\PYG{n}{hello\PYGZus{}world} \PYG{o}{=} \PYG{n+nb}{tuple}\PYG{p}{(}\PYG{p}{(}\PYG{n+nb}{ord}\PYG{p}{(}\PYG{n}{i}\PYG{p}{)} \PYG{k}{for} \PYG{n}{i} \PYG{o+ow}{in} \PYG{l+s}{"}\PYG{l+s}{hello world}\PYG{l+s+se}{\PYGZbs{}n}\PYG{l+s}{"}\PYG{p}{)}\PYG{p}{)}

\PYG{n}{my\PYGZus{}chip} \PYG{o}{=} \PYG{n}{Chip}\PYG{p}{(}
    \PYG{n}{Console}\PYG{p}{(}
        \PYG{n}{Sequence}\PYG{p}{(}\PYG{o}{*}\PYG{n}{hello\PYGZus{}world}\PYG{p}{)}\PYG{p}{,}
    \PYG{p}{)}
\PYG{p}{)}
\end{Verbatim}
\end{classdesc}


\section{Instructions}
\index{chips.instruction (module)}
\hypertarget{module-chips.instruction}{}
\declaremodule[chips.instruction]{}{chips.instruction}
\modulesynopsis{}
The instructions provided here form the basis of the software that can be run inside \emph{Processes}.


\subsection{Instructions Reference}
\index{Block (class in chips)}

\hypertarget{chips.Block}{}\begin{classdesc}{Block}{instructions}
The \emph{Block} statement allows instructions to be nested into a single
statement. Using a \emph{Block} allows a group of instructions to be stored as a
single object.

Example:

\begin{Verbatim}[commandchars=\\\{\}]
\PYG{n}{Initialise} \PYG{o}{=} \PYG{n}{Block}\PYG{p}{(}\PYG{n}{a}\PYG{o}{.}\PYG{n}{set}\PYG{p}{(}\PYG{l+m+mi}{0}\PYG{p}{)}\PYG{p}{,} \PYG{n}{b}\PYG{o}{.}\PYG{n}{set}\PYG{p}{(}\PYG{l+m+mi}{0}\PYG{p}{)}\PYG{p}{,} \PYG{n}{c}\PYG{o}{.}\PYG{n}{set}\PYG{p}{(}\PYG{l+m+mi}{0}\PYG{p}{)}\PYG{p}{)}
\PYG{n}{Process}\PYG{p}{(}\PYG{l+m+mi}{8}\PYG{p}{,}
    \PYG{n}{initialise}\PYG{p}{,}
    \PYG{n}{a}\PYG{o}{.}\PYG{n}{set}\PYG{p}{(}\PYG{n}{a}\PYG{o}{+}\PYG{l+m+mi}{1}\PYG{p}{)}\PYG{p}{,} \PYG{n}{b}\PYG{o}{.}\PYG{n}{set}\PYG{p}{(}\PYG{n}{b}\PYG{o}{+}\PYG{l+m+mi}{1}\PYG{p}{)}\PYG{p}{,} \PYG{n}{c}\PYG{o}{.}\PYG{n}{set}\PYG{p}{(}\PYG{n}{c}\PYG{o}{+}\PYG{l+m+mi}{1}\PYG{p}{)}\PYG{p}{,}
    \PYG{n}{initialise}\PYG{p}{,}
\PYG{p}{)}
\end{Verbatim}
\end{classdesc}
\index{Break (class in chips)}

\hypertarget{chips.Break}{}\begin{classdesc}{Break}{}
The \emph{Break} statement causes the flow of control to immediately exit the loop.

Example:

\begin{Verbatim}[commandchars=\\\{\}]
\PYG{c}{\#equivalent to a While loop}
\PYG{n}{Loop}\PYG{p}{(}
    \PYG{n}{If}\PYG{p}{(}\PYG{n}{condition} \PYG{o}{==} \PYG{l+m+mi}{0}\PYG{p}{,}
        \PYG{n}{Break}\PYG{p}{(}\PYG{p}{)}\PYG{p}{,}
    \PYG{p}{)}\PYG{p}{,}
    \PYG{c}{\#do stuff here}
\PYG{p}{)}\PYG{p}{,}
\end{Verbatim}

Example:

\begin{Verbatim}[commandchars=\\\{\}]
\PYG{c}{\#equivalent to a DoWhile loop}
\PYG{n}{Loop}\PYG{p}{(}
    \PYG{c}{\#do stuff here}
    \PYG{n}{If}\PYG{p}{(}\PYG{n}{condition} \PYG{o}{==} \PYG{l+m+mi}{0}\PYG{p}{,}
        \PYG{n}{Break}\PYG{p}{(}\PYG{p}{)}\PYG{p}{,}
    \PYG{p}{)}\PYG{p}{,}
\PYG{p}{)}\PYG{p}{,}
\end{Verbatim}
\end{classdesc}
\index{Continue (class in chips)}

\hypertarget{chips.Continue}{}\begin{classdesc}{Continue}{}
The \emph{Continue} statement causes the flow of control to immediately jump to
the next iteration of the containing loop.

Example:

\begin{Verbatim}[commandchars=\\\{\}]
\PYG{n}{Process}\PYG{p}{(}\PYG{l+m+mi}{12}\PYG{p}{,}
    \PYG{n}{Loop}\PYG{p}{(}
        \PYG{n}{in\PYGZus{}stream}\PYG{o}{.}\PYG{n}{read}\PYG{p}{(}\PYG{n}{a}\PYG{p}{)}\PYG{p}{,}
        \PYG{n}{If}\PYG{p}{(}\PYG{n}{a}\PYG{o}{\&}\PYG{l+m+mi}{1}\PYG{p}{,}
            \PYG{n}{Continue}\PYG{p}{(}\PYG{p}{)}\PYG{p}{,}
        \PYG{p}{)}\PYG{p}{,}
        \PYG{n}{out\PYGZus{}stream}\PYG{o}{.}\PYG{n}{write}\PYG{p}{(}\PYG{n}{a}\PYG{p}{)}\PYG{p}{,}
    \PYG{p}{)}\PYG{p}{,}
\PYG{p}{)}
\end{Verbatim}
\end{classdesc}
\index{If (class in chips)}

\hypertarget{chips.If}{}\begin{classdesc}{If}{condition, *instructions}
The \emph{If} statement conditionally executes instructions.

The condition of the \emph{If} branch is evaluated, followed by the condition of
each of the optional \emph{ElsIf} branches. If one of the conditions evaluates
to non-zero then the corresponding instructions will be executed. If the
\emph{If} condition, and all of the \emph{ElsIf} conditions evaluate to zero, then
the instructions in the optional \emph{Else} branch will be evaluated.

Example:

\begin{Verbatim}[commandchars=\\\{\}]
\PYG{n}{If}\PYG{p}{(}\PYG{n}{condition}\PYG{p}{,}
    \PYG{c}{\#do something}
\PYG{p}{)}\PYG{o}{.}\PYG{n}{ElsIf}\PYG{p}{(}\PYG{n}{condition}\PYG{p}{,}
    \PYG{c}{\#do something else}
\PYG{p}{)}\PYG{o}{.}\PYG{n}{Else}\PYG{p}{(}
    \PYG{c}{\#if all else fails do this}
\PYG{p}{)}
\end{Verbatim}
\end{classdesc}
\index{Loop (class in chips)}

\hypertarget{chips.Loop}{}\begin{classdesc}{Loop}{*instructions}
The \emph{Loop} statement executes instructions repeatedly.

A \emph{Loop} can be exited using the \emph{Break} instruction. A \emph{Continue}
instruction causes the remainder of instructions in the loop to be skipped.
Execution then repeats from the beginning of the \emph{Loop}.

Example:

\begin{Verbatim}[commandchars=\\\{\}]
\PYG{c}{\#filter filter values over 50 out of a stream}
\PYG{n}{Loop}\PYG{p}{(}
    \PYG{n}{in\PYGZus{}stream}\PYG{o}{.}\PYG{n}{read}\PYG{p}{(}\PYG{n}{a}\PYG{p}{)}\PYG{p}{,}
    \PYG{n}{If}\PYG{p}{(}\PYG{n}{a} \PYG{o}{\textgreater{}} \PYG{l+m+mi}{50}\PYG{p}{,} \PYG{n}{Continue}\PYG{p}{(}\PYG{p}{)}\PYG{p}{)}\PYG{p}{,}
    \PYG{n}{out\PYGZus{}stream}\PYG{o}{.}\PYG{n}{write}\PYG{p}{(}\PYG{n}{a}\PYG{p}{)}\PYG{p}{,}
\PYG{p}{)}\PYG{p}{,}
\end{Verbatim}

Example:

\begin{Verbatim}[commandchars=\\\{\}]
\PYG{c}{\#initialise an array}
\PYG{n}{Loop}\PYG{p}{(}
    \PYG{n}{If}\PYG{p}{(}\PYG{n}{index} \PYG{o}{==} \PYG{l+m+mi}{100}\PYG{p}{,}
        \PYG{n}{Break}\PYG{p}{(}\PYG{p}{)}\PYG{p}{,}
    \PYG{p}{)}\PYG{p}{,}
    \PYG{n}{myarray}\PYG{o}{.}\PYG{n}{write}\PYG{p}{(}\PYG{n}{index}\PYG{p}{,} \PYG{l+m+mi}{0}\PYG{p}{)}\PYG{p}{,}
\PYG{p}{)}\PYG{p}{,}
\end{Verbatim}
\end{classdesc}
\index{Value (class in chips)}

\hypertarget{chips.Value}{}\begin{classdesc}{Value}{expression}
The \emph{Value} statement gives a value to the surrounding \emph{Evaluate}
construct.

An \emph{Evaluate}  expression allows a block of statements to be used as an
expression. When a \emph{Value} is encountered, the supplied expression becomes
the value of the whole evaluate statement.

Example:

\begin{Verbatim}[commandchars=\\\{\}]
\PYG{c}{\#provide a And expression similar to Pythons and expression}
\PYG{k}{def} \PYG{n+nf}{LogicalAnd}\PYG{p}{(}\PYG{n}{a}\PYG{p}{,} \PYG{n}{b}\PYG{p}{)}\PYG{p}{:}
    \PYG{k}{return} \PYG{n}{Evaluate}\PYG{p}{(}
        \PYG{n}{If}\PYG{p}{(}\PYG{n}{a}\PYG{p}{,}
            \PYG{n}{Value}\PYG{p}{(}\PYG{n}{b}\PYG{p}{)}\PYG{p}{,}
        \PYG{p}{)}\PYG{o}{.}\PYG{n}{Else}\PYG{p}{(}
            \PYG{l+m+mi}{0}\PYG{p}{,}
        \PYG{p}{)}\PYG{p}{,}
    \PYG{p}{)}
\end{Verbatim}
\end{classdesc}
\index{WaitUs (class in chips)}

\hypertarget{chips.WaitUs}{}\begin{classdesc}{WaitUs}{}
\emph{WaitUs} causes execution to halt until the next tick of the microsecond
timer.

In practice, this means that the process is stalled for less than 1
microsecond. This behaviour is useful when implementing a real-time
counter function because the execution time of statements does not affect
the time between \emph{WaitUs} statements (Providing the statements do not take
more than 1 microsecond to execute of course!).

Example:

\begin{Verbatim}[commandchars=\\\{\}]
\PYG{n}{seconds} \PYG{o}{=} \PYG{n}{Variable}\PYG{p}{(}\PYG{l+m+mi}{0}\PYG{p}{)}
\PYG{n}{count} \PYG{o}{=} \PYG{n}{Variable}\PYG{p}{(}\PYG{l+m+mi}{0}\PYG{p}{)}
\PYG{n}{Process}\PYG{p}{(}\PYG{l+m+mi}{12}\PYG{p}{,}
    \PYG{n}{seconds}\PYG{o}{.}\PYG{n}{set}\PYG{p}{(}\PYG{l+m+mi}{0}\PYG{p}{)}\PYG{p}{,}
    \PYG{n}{Loop}\PYG{p}{(}
        \PYG{n}{count}\PYG{o}{.}\PYG{n}{set}\PYG{p}{(}\PYG{l+m+mi}{1000}\PYG{p}{)}\PYG{p}{,}
        \PYG{n}{While}\PYG{p}{(}\PYG{n}{count}\PYG{p}{,}
            \PYG{n}{WaitUs}\PYG{p}{(}\PYG{p}{)}\PYG{p}{,}
            \PYG{n}{count}\PYG{o}{.}\PYG{n}{set}\PYG{p}{(}\PYG{n}{count}\PYG{o}{-}\PYG{l+m+mi}{1}\PYG{p}{)}\PYG{p}{,}
        \PYG{p}{)}\PYG{p}{,}
        \PYG{n}{seconds}\PYG{o}{.}\PYG{n}{set}\PYG{p}{(}\PYG{n}{seconds} \PYG{o}{+} \PYG{l+m+mi}{1}\PYG{p}{)}\PYG{p}{,}
        \PYG{n}{out\PYGZus{}stream}\PYG{o}{.}\PYG{n}{write}\PYG{p}{(}\PYG{n}{seconds}\PYG{p}{)}\PYG{p}{,}
    \PYG{p}{)}\PYG{p}{,}
\PYG{p}{)}
\end{Verbatim}
\end{classdesc}
\index{While (class in chips)}

\hypertarget{chips.While}{}\begin{classdesc}{While}{}\end{classdesc}
\index{Scan (class in chips)}

\hypertarget{chips.Scan}{}\begin{classdesc}{Scan}{stream, variable}\end{classdesc}
\index{Print (class in chips)}

\hypertarget{chips.Print}{}\begin{classdesc}{Print}{stream, exp, minimum\_number\_of\_digits=None}\end{classdesc}
\index{Evaluate (class in chips)}

\hypertarget{chips.Evaluate}{}\begin{classdesc}{Evaluate}{*instructions}\end{classdesc}

\resetcurrentobjects
\hypertarget{--doc-automatic\_code\_generation/index}{}

\chapter{Automatic Code Generation}


\section{VHDL Code Generation}
\index{chips.VHDL\_plugin (module)}
\hypertarget{module-chips.VHDL\_plugin}{}
\declaremodule[chips.VHDLplugin]{}{chips.VHDL\_plugin}
\modulesynopsis{}
VHDL Code Generation for streams library


\section{C++ Code Generation}
\index{chips.cpp\_plugin (module)}
\hypertarget{module-chips.cpp\_plugin}{}
\declaremodule[chips.cppplugin]{}{chips.cpp\_plugin}
\modulesynopsis{}
C++ code generator for streams library


\section{Visualisation Code Generation}
\index{chips.visual\_plugin (module)}
\hypertarget{module-chips.visual\_plugin}{}
\declaremodule[chips.visualplugin]{}{chips.visual\_plugin}
\modulesynopsis{}
Visualisation for streams library

\resetcurrentobjects
\hypertarget{--doc-ip\_library/index}{}

\chapter{IP library}
\index{chips.ip (module)}
\hypertarget{module-chips.ip}{}
\declaremodule[chips.ip]{}{chips.ip}
\modulesynopsis{}
\resetcurrentobjects
\hypertarget{--doc-extending\_chips/index}{}

\chapter{Extending the Chips Library}


\chapter{Indices and tables}
\begin{itemize}
\item {} 
\emph{Index}

\item {} 
\emph{Module Index}

\item {} 
\emph{Search Page}

\end{itemize}


\renewcommand{\indexname}{Module Index}
\printmodindex
\renewcommand{\indexname}{Index}
\printindex
\end{document}
