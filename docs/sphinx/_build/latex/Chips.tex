% Generated by Sphinx.
\documentclass[letterpaper,10pt,english]{manual}
\usepackage[utf8]{inputenc}
\usepackage[T1]{fontenc}
\usepackage{babel}
\usepackage{times}
\usepackage[Bjarne]{fncychap}
\usepackage{longtable}
\usepackage{sphinx}


\title{Chips Documentation}
\date{March 19, 2011}
\release{0.1}
\author{Jonathan P Dawson}
\newcommand{\sphinxlogo}{}
\renewcommand{\releasename}{Release}
\makeindex
\makemodindex

\makeatletter
\def\PYG@reset{\let\PYG@it=\relax \let\PYG@bf=\relax%
    \let\PYG@ul=\relax \let\PYG@tc=\relax%
    \let\PYG@bc=\relax \let\PYG@ff=\relax}
\def\PYG@tok#1{\csname PYG@tok@#1\endcsname}
\def\PYG@toks#1+{\ifx\relax#1\empty\else%
    \PYG@tok{#1}\expandafter\PYG@toks\fi}
\def\PYG@do#1{\PYG@bc{\PYG@tc{\PYG@ul{%
    \PYG@it{\PYG@bf{\PYG@ff{#1}}}}}}}
\def\PYG#1#2{\PYG@reset\PYG@toks#1+\relax+\PYG@do{#2}}

\def\PYG@tok@gd{\def\PYG@tc##1{\textcolor[rgb]{0.63,0.00,0.00}{##1}}}
\def\PYG@tok@gu{\let\PYG@bf=\textbf\def\PYG@tc##1{\textcolor[rgb]{0.50,0.00,0.50}{##1}}}
\def\PYG@tok@gt{\def\PYG@tc##1{\textcolor[rgb]{0.00,0.25,0.82}{##1}}}
\def\PYG@tok@gs{\let\PYG@bf=\textbf}
\def\PYG@tok@gr{\def\PYG@tc##1{\textcolor[rgb]{1.00,0.00,0.00}{##1}}}
\def\PYG@tok@cm{\let\PYG@it=\textit\def\PYG@tc##1{\textcolor[rgb]{0.25,0.50,0.56}{##1}}}
\def\PYG@tok@vg{\def\PYG@tc##1{\textcolor[rgb]{0.73,0.38,0.84}{##1}}}
\def\PYG@tok@m{\def\PYG@tc##1{\textcolor[rgb]{0.13,0.50,0.31}{##1}}}
\def\PYG@tok@mh{\def\PYG@tc##1{\textcolor[rgb]{0.13,0.50,0.31}{##1}}}
\def\PYG@tok@cs{\def\PYG@tc##1{\textcolor[rgb]{0.25,0.50,0.56}{##1}}\def\PYG@bc##1{\colorbox[rgb]{1.00,0.94,0.94}{##1}}}
\def\PYG@tok@ge{\let\PYG@it=\textit}
\def\PYG@tok@vc{\def\PYG@tc##1{\textcolor[rgb]{0.73,0.38,0.84}{##1}}}
\def\PYG@tok@il{\def\PYG@tc##1{\textcolor[rgb]{0.13,0.50,0.31}{##1}}}
\def\PYG@tok@go{\def\PYG@tc##1{\textcolor[rgb]{0.19,0.19,0.19}{##1}}}
\def\PYG@tok@cp{\def\PYG@tc##1{\textcolor[rgb]{0.00,0.44,0.13}{##1}}}
\def\PYG@tok@gi{\def\PYG@tc##1{\textcolor[rgb]{0.00,0.63,0.00}{##1}}}
\def\PYG@tok@gh{\let\PYG@bf=\textbf\def\PYG@tc##1{\textcolor[rgb]{0.00,0.00,0.50}{##1}}}
\def\PYG@tok@ni{\let\PYG@bf=\textbf\def\PYG@tc##1{\textcolor[rgb]{0.84,0.33,0.22}{##1}}}
\def\PYG@tok@nl{\let\PYG@bf=\textbf\def\PYG@tc##1{\textcolor[rgb]{0.00,0.13,0.44}{##1}}}
\def\PYG@tok@nn{\let\PYG@bf=\textbf\def\PYG@tc##1{\textcolor[rgb]{0.05,0.52,0.71}{##1}}}
\def\PYG@tok@no{\def\PYG@tc##1{\textcolor[rgb]{0.38,0.68,0.84}{##1}}}
\def\PYG@tok@na{\def\PYG@tc##1{\textcolor[rgb]{0.25,0.44,0.63}{##1}}}
\def\PYG@tok@nb{\def\PYG@tc##1{\textcolor[rgb]{0.00,0.44,0.13}{##1}}}
\def\PYG@tok@nc{\let\PYG@bf=\textbf\def\PYG@tc##1{\textcolor[rgb]{0.05,0.52,0.71}{##1}}}
\def\PYG@tok@nd{\let\PYG@bf=\textbf\def\PYG@tc##1{\textcolor[rgb]{0.33,0.33,0.33}{##1}}}
\def\PYG@tok@ne{\def\PYG@tc##1{\textcolor[rgb]{0.00,0.44,0.13}{##1}}}
\def\PYG@tok@nf{\def\PYG@tc##1{\textcolor[rgb]{0.02,0.16,0.49}{##1}}}
\def\PYG@tok@si{\let\PYG@it=\textit\def\PYG@tc##1{\textcolor[rgb]{0.44,0.63,0.82}{##1}}}
\def\PYG@tok@s2{\def\PYG@tc##1{\textcolor[rgb]{0.25,0.44,0.63}{##1}}}
\def\PYG@tok@vi{\def\PYG@tc##1{\textcolor[rgb]{0.73,0.38,0.84}{##1}}}
\def\PYG@tok@nt{\let\PYG@bf=\textbf\def\PYG@tc##1{\textcolor[rgb]{0.02,0.16,0.45}{##1}}}
\def\PYG@tok@nv{\def\PYG@tc##1{\textcolor[rgb]{0.73,0.38,0.84}{##1}}}
\def\PYG@tok@s1{\def\PYG@tc##1{\textcolor[rgb]{0.25,0.44,0.63}{##1}}}
\def\PYG@tok@gp{\let\PYG@bf=\textbf\def\PYG@tc##1{\textcolor[rgb]{0.78,0.36,0.04}{##1}}}
\def\PYG@tok@sh{\def\PYG@tc##1{\textcolor[rgb]{0.25,0.44,0.63}{##1}}}
\def\PYG@tok@ow{\let\PYG@bf=\textbf\def\PYG@tc##1{\textcolor[rgb]{0.00,0.44,0.13}{##1}}}
\def\PYG@tok@sx{\def\PYG@tc##1{\textcolor[rgb]{0.78,0.36,0.04}{##1}}}
\def\PYG@tok@bp{\def\PYG@tc##1{\textcolor[rgb]{0.00,0.44,0.13}{##1}}}
\def\PYG@tok@c1{\let\PYG@it=\textit\def\PYG@tc##1{\textcolor[rgb]{0.25,0.50,0.56}{##1}}}
\def\PYG@tok@kc{\let\PYG@bf=\textbf\def\PYG@tc##1{\textcolor[rgb]{0.00,0.44,0.13}{##1}}}
\def\PYG@tok@c{\let\PYG@it=\textit\def\PYG@tc##1{\textcolor[rgb]{0.25,0.50,0.56}{##1}}}
\def\PYG@tok@mf{\def\PYG@tc##1{\textcolor[rgb]{0.13,0.50,0.31}{##1}}}
\def\PYG@tok@err{\def\PYG@bc##1{\fcolorbox[rgb]{1.00,0.00,0.00}{1,1,1}{##1}}}
\def\PYG@tok@kd{\let\PYG@bf=\textbf\def\PYG@tc##1{\textcolor[rgb]{0.00,0.44,0.13}{##1}}}
\def\PYG@tok@ss{\def\PYG@tc##1{\textcolor[rgb]{0.32,0.47,0.09}{##1}}}
\def\PYG@tok@sr{\def\PYG@tc##1{\textcolor[rgb]{0.14,0.33,0.53}{##1}}}
\def\PYG@tok@mo{\def\PYG@tc##1{\textcolor[rgb]{0.13,0.50,0.31}{##1}}}
\def\PYG@tok@mi{\def\PYG@tc##1{\textcolor[rgb]{0.13,0.50,0.31}{##1}}}
\def\PYG@tok@kn{\let\PYG@bf=\textbf\def\PYG@tc##1{\textcolor[rgb]{0.00,0.44,0.13}{##1}}}
\def\PYG@tok@o{\def\PYG@tc##1{\textcolor[rgb]{0.40,0.40,0.40}{##1}}}
\def\PYG@tok@kr{\let\PYG@bf=\textbf\def\PYG@tc##1{\textcolor[rgb]{0.00,0.44,0.13}{##1}}}
\def\PYG@tok@s{\def\PYG@tc##1{\textcolor[rgb]{0.25,0.44,0.63}{##1}}}
\def\PYG@tok@kp{\def\PYG@tc##1{\textcolor[rgb]{0.00,0.44,0.13}{##1}}}
\def\PYG@tok@w{\def\PYG@tc##1{\textcolor[rgb]{0.73,0.73,0.73}{##1}}}
\def\PYG@tok@kt{\def\PYG@tc##1{\textcolor[rgb]{0.56,0.13,0.00}{##1}}}
\def\PYG@tok@sc{\def\PYG@tc##1{\textcolor[rgb]{0.25,0.44,0.63}{##1}}}
\def\PYG@tok@sb{\def\PYG@tc##1{\textcolor[rgb]{0.25,0.44,0.63}{##1}}}
\def\PYG@tok@k{\let\PYG@bf=\textbf\def\PYG@tc##1{\textcolor[rgb]{0.00,0.44,0.13}{##1}}}
\def\PYG@tok@se{\let\PYG@bf=\textbf\def\PYG@tc##1{\textcolor[rgb]{0.25,0.44,0.63}{##1}}}
\def\PYG@tok@sd{\let\PYG@it=\textit\def\PYG@tc##1{\textcolor[rgb]{0.25,0.44,0.63}{##1}}}

\def\PYGZbs{\char`\\}
\def\PYGZus{\char`\_}
\def\PYGZob{\char`\{}
\def\PYGZcb{\char`\}}
\def\PYGZca{\char`\^}
% for compatibility with earlier versions
\def\PYGZat{@}
\def\PYGZlb{[}
\def\PYGZrb{]}
\makeatother

\begin{document}

\maketitle
\tableofcontents
\hypertarget{--doc-index}{}


Contents:

\resetcurrentobjects
\hypertarget{--doc-language\_reference/index}{}

\chapter{Chips Language Reference manual}


\section{Chip}
\index{streams (module)}
\hypertarget{module-streams}{}
\declaremodule[streams]{}{streams}
\modulesynopsis{}
A Stream based concurrent programming library for embedded systems
\index{System (class in streams)}

\hypertarget{streams.System}{}\begin{classdesc}{System}{*args}
A Chip device containing streams, sinks and processes.

Typically a System is used to describe a single device. You need to provide
the System object with a list of all the sinks (devie outputs). You don't
need to include any process, variables or streams. By analysing the sinks,
the system can work out which processes and streams need to be included in
the device.

Example:

\begin{Verbatim}[commandchars=\\\{\}]
\PYG{n}{switches} \PYG{o}{=} \PYG{n}{InPort}\PYG{p}{(}\PYG{l+s}{"}\PYG{l+s}{SWITCHES}\PYG{l+s}{"}\PYG{p}{,} \PYG{l+m+mi}{8}\PYG{p}{)}
\PYG{n}{serial\PYGZus{}in} \PYG{o}{=} \PYG{n}{SerialIn}\PYG{p}{(}\PYG{l+s}{"}\PYG{l+s}{RX}\PYG{l+s}{"}\PYG{p}{)}
\PYG{n}{leds} \PYG{o}{=} \PYG{n}{OutPort}\PYG{p}{(}\PYG{n}{switches}\PYG{p}{,} \PYG{l+s}{"}\PYG{l+s}{LEDS}\PYG{l+s}{"}\PYG{p}{)}
\PYG{n}{serial\PYGZus{}out} \PYG{o}{=} \PYG{n}{SerialOut}\PYG{p}{(}\PYG{l+s}{"}\PYG{l+s}{TX}\PYG{l+s}{"}\PYG{p}{,} \PYG{n}{serial\PYGZus{}in}\PYG{p}{)}

\PYG{c}{\#We need to tell the *System* that *leds* and *serial\PYGZus{}out* are part of}
\PYG{c}{\#the device. The *System* can work out for itself that *switches* and}
\PYG{c}{\#*serial\PYGZus{}in* are part of the device.}

\PYG{n}{s} \PYG{o}{=} \PYG{n}{System}\PYG{p}{(}
    \PYG{n}{leds}\PYG{p}{,}
    \PYG{n}{serial\PYGZus{}out}\PYG{p}{,}
\PYG{p}{)}

\PYG{n}{s}\PYG{o}{.}\PYG{n}{write\PYGZus{}code}\PYG{p}{(}\PYG{n}{plugin}\PYG{p}{)}
\end{Verbatim}
\end{classdesc}


\section{Processes}
\index{Process (class in streams)}

\hypertarget{streams.Process}{}\begin{classdesc}{Process}{bits, *instructions}\end{classdesc}
\index{Variable (class in streams)}

\hypertarget{streams.Variable}{}\begin{classdesc}{Variable}{initial}\end{classdesc}


\section{Streams}
\index{Array (class in streams)}

\hypertarget{streams.Array}{}\begin{classdesc}{Array}{address\_in, data\_in, address\_out, depth}\end{classdesc}
\index{Counter (class in streams)}

\hypertarget{streams.Counter}{}\begin{classdesc}{Counter}{start, stop, step}
A Stream which yields numbers from \emph{start} to \emph{stop} in \emph{step} increments.

A \emph{Counter} is a versatile, and commonly used construct in device design,
they can be used to number samples, index memories and so on.

Example:

\begin{Verbatim}[commandchars=\\\{\}]
\PYG{n}{Counter}\PYG{p}{(}\PYG{l+m+mi}{0}\PYG{p}{,} \PYG{l+m+mi}{10}\PYG{p}{,} \PYG{l+m+mi}{2}\PYG{p}{)} \PYG{c}{\# --\textgreater{} 0 2 4 6 8 10 0 \PYGZbs{}..}

\PYG{n}{Counter}\PYG{p}{(}\PYG{l+m+mi}{10}\PYG{p}{,} \PYG{l+m+mi}{0}\PYG{p}{,} \PYG{o}{-}\PYG{l+m+mi}{2}\PYG{p}{)} \PYG{c}{\# --\textgreater{} 10 8 7 6 4 2 0 10 \PYGZbs{}..}
\end{Verbatim}
\end{classdesc}
\index{Decoupler (class in streams)}

\hypertarget{streams.Decoupler}{}\begin{classdesc}{Decoupler}{source}\end{classdesc}
\index{Resizer (class in streams)}

\hypertarget{streams.Resizer}{}\begin{classdesc}{Resizer}{source, bits}\end{classdesc}
\index{Lookup (class in streams)}

\hypertarget{streams.Lookup}{}\begin{classdesc}{Lookup}{source, *args}\end{classdesc}
\index{Fifo (class in streams)}

\hypertarget{streams.Fifo}{}\begin{classdesc}{Fifo}{data\_in, depth}\end{classdesc}
\index{Repeater (class in streams)}

\hypertarget{streams.Repeater}{}\begin{classdesc}{Repeater}{value}
A stream which repeatedly yields the specified \emph{value}.

The \emph{Repeater} stream is one of the most fundamental streams available.

The width of the stream in bits is calculated automatically. The smallest
number of bits that can represent \emph{value} in twos-complement format will be
used.

Examples:

\begin{Verbatim}[commandchars=\\\{\}]
\PYG{n}{Repeater}\PYG{p}{(}\PYG{l+m+mi}{5}\PYG{p}{)} \PYG{c}{\#--\textgreater{} 5 5 5 5 \PYGZbs{}..}
\PYG{c}{\#creates a 4 bit stream.}

\PYG{n}{Repeater}\PYG{p}{(}\PYG{l+m+mi}{10}\PYG{p}{)} \PYG{c}{\#--\textgreater{} 10 10 10 10 \PYGZbs{}..}
\PYG{c}{\#creates a 5 bit stream.}

\PYG{n}{Repeater}\PYG{p}{(}\PYG{l+m+mi}{5}\PYG{p}{)}\PYG{o}{*}\PYG{l+m+mi}{2} \PYG{c}{\#--\textgreater{} 10 10 10 10 \PYGZbs{}..}
\PYG{c}{\#This is shothand for: Repeater(5)*Repeater(2)}
\end{Verbatim}
\end{classdesc}
\index{Sequence (class in streams)}

\hypertarget{streams.Sequence}{}\begin{classdesc}{Sequence}{}\end{classdesc}
\index{Stimulus (class in streams)}

\hypertarget{streams.Stimulus}{}\begin{classdesc}{Stimulus}{bits}
A Stream that allows a Python iterable to be used as a stream.

A Simulus stream allows a transparent method to pass data from the Python
envrinment into the simulation environment. The sequence object is set at
run time using the set\_simulation\_data() method. The sequence object can be
any iterable Python sequence such as a list, tuple, or even a generator.

Example:

\begin{Verbatim}[commandchars=\\\{\}]
\PYG{k+kn}{import} \PYG{n+nn}{PIL}

\PYG{n}{picture} \PYG{o}{=} \PYG{n}{Stimulus}\PYG{p}{(}\PYG{p}{)}
\PYG{n}{s} \PYG{o}{=} \PYG{n}{System}\PYG{p}{(}\PYG{n}{Console}\PYG{p}{(}\PYG{n}{Printer}\PYG{p}{(}\PYG{n}{picture}\PYG{p}{)}\PYG{p}{)}\PYG{p}{)}

\PYG{n}{im} \PYG{o}{=} \PYG{n}{PIL}\PYG{o}{.}\PYG{n}{open}\PYG{p}{(}\PYG{l+s}{"}\PYG{l+s}{test.bmp}\PYG{l+s}{"}\PYG{p}{)}
\PYG{n}{image\PYGZus{}data} \PYG{o}{=} \PYG{n+nb}{list}\PYG{p}{(}\PYG{n}{im}\PYG{o}{.}\PYG{n}{getdata}\PYG{p}{(}\PYG{p}{)}\PYG{p}{)}
\PYG{n}{picture}\PYG{o}{.}\PYG{n}{set\PYGZus{}simulation\PYGZus{}data}\PYG{p}{(}\PYG{n}{image\PYGZus{}data}\PYG{p}{)}

\PYG{n}{picture}\PYG{o}{.}\PYG{n}{reset}\PYG{p}{(}\PYG{p}{)}
\PYG{n}{picture}\PYG{o}{.}\PYG{n}{execute}\PYG{p}{(}\PYG{l+m+mi}{1000}\PYG{p}{)}
\end{Verbatim}
\end{classdesc}
\index{InPort (class in streams)}

\hypertarget{streams.InPort}{}\begin{classdesc}{InPort}{name, bits}
A device input port stream.

An \emph{InPort} allows a port pins of the target device to be used as a data
stream.  There is no handshaking on the input port. The port pins are
sampled at the point when data is transfered by the stream.  When
implemented in VHDL, the \emph{InPort} provides double registers on the port
pins to synchronise data to the local clock domain.

Since it is not possible to determine the width of the strean in bits
automatically, this must be specified using the \emph{bits} argument.

The \emph{name} parameter allows a string to be associated with the input port.
In a VHDL implementation, \emph{name} will be used as the port name in the
top level entity.

Example:

\begin{Verbatim}[commandchars=\\\{\}]
\PYG{n}{dip\PYGZus{}switches} \PYG{o}{=} \PYG{n}{Inport}\PYG{p}{(}\PYG{l+s}{"}\PYG{l+s}{dip\PYGZus{}switches}\PYG{l+s}{"}\PYG{p}{,} \PYG{l+m+mi}{8}\PYG{p}{)} 
\PYG{n}{s} \PYG{o}{=} \PYG{n}{System}\PYG{p}{(}\PYG{n}{SerialOut}\PYG{p}{(}\PYG{n}{Printer}\PYG{p}{(}\PYG{n}{dip\PYGZus{}switches}\PYG{p}{)}\PYG{p}{)}\PYG{p}{)}
\end{Verbatim}
\end{classdesc}
\index{SerialIn (class in streams)}

\hypertarget{streams.SerialIn}{}\begin{classdesc}{SerialIn}{name='RX', clock\_rate=50000000, baud\_rate=115200}
A \emph{SerialIn} yields 8-bit data from a serial uart input.
\end{classdesc}
\index{Output (class in streams)}

\hypertarget{streams.Output}{}\begin{classdesc}{Output}{}\end{classdesc}
\index{Printer (class in streams)}

\hypertarget{streams.Printer}{}\begin{classdesc}{Printer}{source}\end{classdesc}
\index{HexPrinter (class in streams)}

\hypertarget{streams.HexPrinter}{}\begin{classdesc}{HexPrinter}{source}\end{classdesc}
\index{Scanner (class in streams)}

\hypertarget{streams.Scanner}{}\begin{classdesc}{Scanner}{}\end{classdesc}


\section{Sinks}
\index{Response (class in streams)}

\hypertarget{streams.Response}{}\begin{classdesc}{Response}{a}
A Response block allows data to be read from a stream in the python 
design environment. A similar interface can be used in native python
simulations and also co-simulations using external tools.
\end{classdesc}
\index{OutPort (class in streams)}

\hypertarget{streams.OutPort}{}\begin{classdesc}{OutPort}{a, name}\end{classdesc}
\index{SerialOut (class in streams)}

\hypertarget{streams.SerialOut}{}\begin{classdesc}{SerialOut}{a, name='TX', clock\_rate=50000000, baud\_rate=115200}\end{classdesc}
\index{Asserter (class in streams)}

\hypertarget{streams.Asserter}{}\begin{classdesc}{Asserter}{a}\end{classdesc}
\index{Console (class in streams)}

\hypertarget{streams.Console}{}\begin{classdesc}{Console}{a}\end{classdesc}


\section{Instructions}
\index{Block (class in streams)}

\hypertarget{streams.Block}{}\begin{classdesc}{Block}{instructions}
The \emph{Block} statement allows instructions to be nested into a single
statement. Using a \emph{Block} allows a group of instructions to be stored as a
single object.

Example:

\begin{Verbatim}[commandchars=\\\{\}]
\PYG{n}{intialise} \PYG{o}{=} \PYG{n}{Block}\PYG{p}{(}\PYG{n}{a}\PYG{o}{.}\PYG{n}{set}\PYG{p}{(}\PYG{l+m+mi}{0}\PYG{p}{)}\PYG{p}{,} \PYG{n}{b}\PYG{o}{.}\PYG{n}{set}\PYG{p}{(}\PYG{l+m+mi}{0}\PYG{p}{)}\PYG{p}{,} \PYG{n}{c}\PYG{o}{.}\PYG{n}{set}\PYG{p}{(}\PYG{l+m+mi}{0}\PYG{p}{)}\PYG{p}{)}
\PYG{n}{Process}\PYG{p}{(}\PYG{l+m+mi}{8}\PYG{p}{,}
    \PYG{n}{initialise}\PYG{p}{,}
    \PYG{n}{a}\PYG{o}{.}\PYG{n}{set}\PYG{p}{(}\PYG{n}{a}\PYG{o}{+}\PYG{l+m+mi}{1}\PYG{p}{)}\PYG{p}{,} \PYG{n}{b}\PYG{o}{.}\PYG{n}{set}\PYG{p}{(}\PYG{n}{b}\PYG{o}{+}\PYG{l+m+mi}{1}\PYG{p}{)}\PYG{p}{,} \PYG{n}{c}\PYG{o}{.}\PYG{n}{set}\PYG{p}{(}\PYG{n}{c}\PYG{o}{+}\PYG{l+m+mi}{1}\PYG{p}{)}\PYG{p}{,}
    \PYG{n}{initialise}\PYG{p}{,}
\PYG{p}{)}
\end{Verbatim}
\end{classdesc}
\index{Break (class in streams)}

\hypertarget{streams.Break}{}\begin{classdesc}{Break}{}
The \emph{Break} statement causes the flow of control to immediately exit the loop.

Example:

\begin{Verbatim}[commandchars=\\\{\}]
\PYG{c}{\#equivilent to a While loop}
\PYG{n}{Loop}\PYG{p}{(}
    \PYG{n}{If}\PYG{p}{(}\PYG{n}{condition} \PYG{o}{==} \PYG{l+m+mi}{0}\PYG{p}{,}
        \PYG{n}{Break}\PYG{p}{(}\PYG{p}{)}\PYG{p}{,}
    \PYG{p}{)}\PYG{p}{,}
    \PYG{c}{\#do stuff here}
\PYG{p}{)}\PYG{p}{,}
\end{Verbatim}

Example:

\begin{Verbatim}[commandchars=\\\{\}]
\PYG{c}{\#equivilent to a DoWhile loop}
\PYG{n}{Loop}\PYG{p}{(}
    \PYG{c}{\#do stuff here}
    \PYG{n}{If}\PYG{p}{(}\PYG{n}{condition} \PYG{o}{==} \PYG{l+m+mi}{0}\PYG{p}{,}
        \PYG{n}{Break}\PYG{p}{(}\PYG{p}{)}\PYG{p}{,}
    \PYG{p}{)}\PYG{p}{,}
\PYG{p}{)}\PYG{p}{,}
\end{Verbatim}
\end{classdesc}
\index{Continue (class in streams)}

\hypertarget{streams.Continue}{}\begin{classdesc}{Continue}{}
The \emph{Continue} statement causes the flow of control to immediately jump to
the next iteration of the contatining loop.

Example:

\begin{Verbatim}[commandchars=\\\{\}]
\PYG{n}{Process}\PYG{p}{(}\PYG{l+m+mi}{12}\PYG{p}{,}
    \PYG{n}{Loop}\PYG{p}{(}
        \PYG{n}{in\PYGZus{}stream}\PYG{o}{.}\PYG{n}{read}\PYG{p}{(}\PYG{n}{a}\PYG{p}{)}\PYG{p}{,}
        \PYG{n}{If}\PYG{p}{(}\PYG{n}{a}\PYG{o}{\&}\PYG{l+m+mi}{1}\PYG{p}{,}
            \PYG{n}{Continue}\PYG{p}{(}\PYG{p}{)}\PYG{p}{,}
        \PYG{p}{)}\PYG{p}{,}
        \PYG{n}{out\PYGZus{}stream}\PYG{o}{.}\PYG{n}{write}\PYG{p}{(}\PYG{n}{a}\PYG{p}{)}\PYG{p}{,}
    \PYG{p}{)}\PYG{p}{,}
\PYG{p}{)}
\end{Verbatim}
\end{classdesc}
\index{If (class in streams)}

\hypertarget{streams.If}{}\begin{classdesc}{If}{condition, *instructions}
The \emph{If} statement conditionaly executes instructions.

The condition of the \emph{If} branch is evaluated, followed by the condition of
each of the optional \emph{ElsIf} branches. If one of the conditions evaluates
to non-zero then the corresponding instructions will be executed. If the
\emph{If} condition, and all of the \emph{ElsIf} conditions evaluate to zero, then
the instructions in the optional \emph{Else} branch will be evaluated.

Example:

\begin{Verbatim}[commandchars=\\\{\}]
\PYG{n}{If}\PYG{p}{(}\PYG{n}{condition}\PYG{p}{,}
    \PYG{c}{\#do something}
\PYG{p}{)}\PYG{o}{.}\PYG{n}{Elsif}\PYG{p}{(}\PYG{n}{condition}\PYG{p}{,}
    \PYG{c}{\#do something else}
\PYG{p}{)}\PYG{o}{.}\PYG{n}{Else}\PYG{p}{(}
    \PYG{c}{\#if all else fails do this}
\PYG{p}{)}
\end{Verbatim}
\end{classdesc}
\index{Loop (class in streams)}

\hypertarget{streams.Loop}{}\begin{classdesc}{Loop}{*instructions}
The \emph{Loop} statement executes instructions repeatedly.

A \emph{Loop} can be exited using the \emph{Break} instruction. A \emph{Continue}
instruction causes the remainder of intructions in the loop to be skipped.
Execution then repeats from the begining of the \emph{Loop}.

Example:

\begin{Verbatim}[commandchars=\\\{\}]
\PYG{c}{\#filter filter values over 50 out of a stream}
\PYG{n}{Loop}\PYG{p}{(}
    \PYG{n}{in\PYGZus{}stream}\PYG{o}{.}\PYG{n}{read}\PYG{p}{(}\PYG{n}{a}\PYG{p}{)}\PYG{p}{,}
    \PYG{n}{If}\PYG{p}{(}\PYG{n}{a} \PYG{o}{\textgreater{}} \PYG{l+m+mi}{50}\PYG{p}{,} \PYG{n}{Continue}\PYG{p}{(}\PYG{p}{)}\PYG{p}{)}\PYG{p}{,}
    \PYG{n}{out\PYGZus{}stream}\PYG{o}{.}\PYG{n}{write}\PYG{p}{(}\PYG{n}{a}\PYG{p}{)}\PYG{p}{,}
\PYG{p}{)}\PYG{p}{,}
\end{Verbatim}

Example:

\begin{Verbatim}[commandchars=\\\{\}]
\PYG{c}{\#initialise an array}
\PYG{n}{Loop}\PYG{p}{(}
    \PYG{n}{If}\PYG{p}{(}\PYG{n}{index} \PYG{o}{==} \PYG{l+m+mi}{100}\PYG{p}{,}
        \PYG{n}{Break}\PYG{p}{(}\PYG{p}{)}\PYG{p}{,}
    \PYG{p}{)}\PYG{p}{,}
    \PYG{n}{myarray}\PYG{o}{.}\PYG{n}{write}\PYG{p}{(}\PYG{n}{index}\PYG{p}{,} \PYG{l+m+mi}{0}\PYG{p}{)}\PYG{p}{,}
\PYG{p}{)}\PYG{p}{,}
\end{Verbatim}
\end{classdesc}
\index{Value (class in streams)}

\hypertarget{streams.Value}{}\begin{classdesc}{Value}{expression}
The \emph{Value} statement gives a value to the surrounding \emph{Evaluate}
construct.

An \emph{Evaluate}  expression allows a block of statements to be used as an
expression. When a \emph{Value} is encountered, the supplied expression becomes
the value of the whole evaluate statement.

Example:

\begin{Verbatim}[commandchars=\\\{\}]
\PYG{c}{\#provide a And expression similar to Pythons and expression}
\PYG{k}{def} \PYG{n+nf}{LogicalAnd}\PYG{p}{(}\PYG{n}{a}\PYG{p}{,} \PYG{n}{b}\PYG{p}{)}\PYG{p}{:}
    \PYG{k}{return} \PYG{n}{Evaluate}\PYG{p}{(}
        \PYG{n}{If}\PYG{p}{(}\PYG{n}{a}\PYG{p}{,}
            \PYG{n}{Value}\PYG{p}{(}\PYG{n}{b}\PYG{p}{)}\PYG{p}{,}
        \PYG{p}{)}\PYG{o}{.}\PYG{n}{Else}\PYG{p}{(}
            \PYG{l+m+mi}{0}\PYG{p}{,}
        \PYG{p}{)}\PYG{p}{,}
    \PYG{p}{)}
\end{Verbatim}
\end{classdesc}
\index{WaitUs (class in streams)}

\hypertarget{streams.WaitUs}{}\begin{classdesc}{WaitUs}{}
\emph{WaitUs} causes execution to halt until the next tick of the microsecond
timer.

In practice, this means that the the process is stalled for less than 1
microsecond. This behaviour is usefull when implementing a real-time
counter function because the execution time of statements does not affect
the time between \emph{WaitUs} statements (Providing the statements do not take
more than 1 microsecond to execute of course!).

Example:

\begin{Verbatim}[commandchars=\\\{\}]
\PYG{n}{seconds} \PYG{o}{=} \PYG{n}{Variable}\PYG{p}{(}\PYG{l+m+mi}{0}\PYG{p}{)}
\PYG{n}{count} \PYG{o}{=} \PYG{n}{Variable}\PYG{p}{(}\PYG{l+m+mi}{0}\PYG{p}{)}
\PYG{n}{Process}\PYG{p}{(}\PYG{l+m+mi}{12}\PYG{p}{,}
    \PYG{n}{seconds}\PYG{o}{.}\PYG{n}{set}\PYG{p}{(}\PYG{l+m+mi}{0}\PYG{p}{)}\PYG{p}{,}
    \PYG{n}{Loop}\PYG{p}{(}
        \PYG{n}{count}\PYG{o}{.}\PYG{n}{set}\PYG{p}{(}\PYG{l+m+mi}{1000}\PYG{p}{)}\PYG{p}{,}
        \PYG{n}{While}\PYG{p}{(}\PYG{n}{count}\PYG{p}{,}
            \PYG{n}{WaitUs}\PYG{p}{(}\PYG{p}{)}\PYG{p}{,}
            \PYG{n}{count}\PYG{o}{.}\PYG{n}{set}\PYG{p}{(}\PYG{n}{count}\PYG{o}{-}\PYG{l+m+mi}{1}\PYG{p}{)}\PYG{p}{,}
        \PYG{p}{)}\PYG{p}{,}
        \PYG{n}{seconds}\PYG{o}{.}\PYG{n}{set}\PYG{p}{(}\PYG{n}{seconds} \PYG{o}{+} \PYG{l+m+mi}{1}\PYG{p}{)}\PYG{p}{,}
        \PYG{n}{out\PYGZus{}stream}\PYG{o}{.}\PYG{n}{write}\PYG{p}{(}\PYG{n}{seconds}\PYG{p}{)}\PYG{p}{,}
    \PYG{p}{)}\PYG{p}{,}
\PYG{p}{)}
\end{Verbatim}
\end{classdesc}
\index{While (class in streams)}

\hypertarget{streams.While}{}\begin{classdesc}{While}{}\end{classdesc}
\index{Scan (class in streams)}

\hypertarget{streams.Scan}{}\begin{classdesc}{Scan}{stream, variable}\end{classdesc}
\index{Print (class in streams)}

\hypertarget{streams.Print}{}\begin{classdesc}{Print}{stream, exp, minimum\_number\_of\_digits=None}\end{classdesc}
\index{Evaluate (class in streams)}

\hypertarget{streams.Evaluate}{}\begin{classdesc}{Evaluate}{*instructions}\end{classdesc}

\resetcurrentobjects
\hypertarget{--doc-automatic\_code\_generation/index}{}

\chapter{Automatic Code Generation}


\section{VHDL Code Generation}


\section{C++ Code Generation}
\index{streams\_cpp (module)}
\hypertarget{module-streams\_cpp}{}
\declaremodule[streamscpp]{}{streams\_cpp}
\modulesynopsis{}
C++ code generator for streams library


\section{Visualisation Code Generation}


\chapter{Indices and tables}
\begin{itemize}
\item {} 
\emph{Index}

\item {} 
\emph{Module Index}

\item {} 
\emph{Search Page}

\end{itemize}


\renewcommand{\indexname}{Module Index}
\printmodindex
\renewcommand{\indexname}{Index}
\printindex
\end{document}
